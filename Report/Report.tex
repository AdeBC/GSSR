\documentclass[11pt]{article}

    \usepackage[breakable]{tcolorbox}
    \usepackage{parskip} % Stop auto-indenting (to mimic markdown behaviour)
    \usepackage{subcaption}
    \usepackage{iftex}
    \ifPDFTeX
    	\usepackage[T1]{fontenc}
    	\usepackage{mathpazo}
    \else
    	\usepackage{fontspec}
    \fi

    % Basic figure setup, for now with no caption control since it's done
    % automatically by Pandoc (which extracts ![](path) syntax from Markdown).
    \usepackage{graphicx}
    % Maintain compatibility with old templates. Remove in nbconvert 6.0
    % \let\Oldincludegraphics\includegraphics
    % Ensure that by default, figures have no caption (until we provide a
    % proper Figure object with a Caption API and a way to capture that
    % in the conversion process - todo).
    \usepackage{caption}
%    \DeclareCaptionFormat{nocaption}{}
%    \captionsetup{format=nocaption,aboveskip=0pt,belowskip=0pt}

    \usepackage[Export]{adjustbox} % Used to constrain images to a maximum size
    \adjustboxset{max size={0.9\linewidth}{0.9\paperheight}}
    \usepackage{float}
    \floatplacement{figure}{H} % forces figures to be placed at the correct location
    \usepackage{xcolor} % Allow colors to be defined
    \usepackage{enumerate} % Needed for markdown enumerations to work
    \usepackage{geometry} % Used to adjust the document margins
    \usepackage{amsmath} % Equations
    \usepackage{amssymb} % Equations
    \usepackage{textcomp} % defines textquotesingle
    % Hack from http://tex.stackexchange.com/a/47451/13684:
    \AtBeginDocument{%
        \def\PYZsq{\textquotesingle}% Upright quotes in Pygmentized code
    }
    \usepackage{upquote} % Upright quotes for verbatim code
    \usepackage{eurosym} % defines \euro
    \usepackage[mathletters]{ucs} % Extended unicode (utf-8) support
    \usepackage{fancyvrb} % verbatim replacement that allows latex
    \usepackage{grffile} % extends the file name processing of package graphics 
                         % to support a larger range
    \makeatletter % fix for grffile with XeLaTeX
    \def\Gread@@xetex#1{%
      \IfFileExists{"\Gin@base".bb}%
      {\Gread@eps{\Gin@base.bb}}%
      {\Gread@@xetex@aux#1}%
    }
    \makeatother

    % The hyperref package gives us a pdf with properly built
    % internal navigation ('pdf bookmarks' for the table of contents,
    % internal cross-reference links, web links for URLs, etc.)
    \usepackage{hyperref}
    % The default LaTeX title has an obnoxious amount of whitespace. By default,
    % titling removes some of it. It also provides customization options.
    \usepackage{titling}
    \usepackage{longtable} % longtable support required by pandoc >1.10
    \usepackage{booktabs}  % table support for pandoc > 1.12.2
    \usepackage[inline]{enumitem} % IRkernel/repr support (it uses the enumerate* environment)
    \usepackage[normalem]{ulem} % ulem is needed to support strikethroughs (\sout)
                                % normalem makes italics be italics, not underlines
    \usepackage{mathrsfs}
    \usepackage{threeparttable}

    
    % Colors for the hyperref package
    \definecolor{urlcolor}{rgb}{0,.145,.698}
    \definecolor{linkcolor}{rgb}{.71,0.21,0.01}
    \definecolor{citecolor}{rgb}{.12,.54,.11}

    % ANSI colors
    \definecolor{ansi-black}{HTML}{3E424D}
    \definecolor{ansi-black-intense}{HTML}{282C36}
    \definecolor{ansi-red}{HTML}{E75C58}
    \definecolor{ansi-red-intense}{HTML}{B22B31}
    \definecolor{ansi-green}{HTML}{00A250}
    \definecolor{ansi-green-intense}{HTML}{007427}
    \definecolor{ansi-yellow}{HTML}{DDB62B}
    \definecolor{ansi-yellow-intense}{HTML}{B27D12}
    \definecolor{ansi-blue}{HTML}{208FFB}
    \definecolor{ansi-blue-intense}{HTML}{0065CA}
    \definecolor{ansi-magenta}{HTML}{D160C4}
    \definecolor{ansi-magenta-intense}{HTML}{A03196}
    \definecolor{ansi-cyan}{HTML}{60C6C8}
    \definecolor{ansi-cyan-intense}{HTML}{258F8F}
    \definecolor{ansi-white}{HTML}{C5C1B4}
    \definecolor{ansi-white-intense}{HTML}{A1A6B2}
    \definecolor{ansi-default-inverse-fg}{HTML}{FFFFFF}
    \definecolor{ansi-default-inverse-bg}{HTML}{000000}

    % commands and environments needed by pandoc snippets
    % extracted from the output of `pandoc -s`
    \providecommand{\tightlist}{%
      \setlength{\itemsep}{0pt}\setlength{\parskip}{0pt}}
    \DefineVerbatimEnvironment{Highlighting}{Verbatim}{commandchars=\\\{\},fontsize=\small}
    % Add ',fontsize=\small' for more characters per line
    \newenvironment{Shaded}{}{}
    \newcommand{\KeywordTok}[1]{\textcolor[rgb]{0.00,0.44,0.13}{\textbf{{#1}}}}
    \newcommand{\DataTypeTok}[1]{\textcolor[rgb]{0.56,0.13,0.00}{{#1}}}
    \newcommand{\DecValTok}[1]{\textcolor[rgb]{0.25,0.63,0.44}{{#1}}}
    \newcommand{\BaseNTok}[1]{\textcolor[rgb]{0.25,0.63,0.44}{{#1}}}
    \newcommand{\FloatTok}[1]{\textcolor[rgb]{0.25,0.63,0.44}{{#1}}}
    \newcommand{\CharTok}[1]{\textcolor[rgb]{0.25,0.44,0.63}{{#1}}}
    \newcommand{\StringTok}[1]{\textcolor[rgb]{0.25,0.44,0.63}{{#1}}}
    \newcommand{\CommentTok}[1]{\textcolor[rgb]{0.38,0.63,0.69}{\textit{{#1}}}}
    \newcommand{\OtherTok}[1]{\textcolor[rgb]{0.00,0.44,0.13}{{#1}}}
    \newcommand{\AlertTok}[1]{\textcolor[rgb]{1.00,0.00,0.00}{\textbf{{#1}}}}
    \newcommand{\FunctionTok}[1]{\textcolor[rgb]{0.02,0.16,0.49}{{#1}}}
    \newcommand{\RegionMarkerTok}[1]{{#1}}
    \newcommand{\ErrorTok}[1]{\textcolor[rgb]{1.00,0.00,0.00}{\textbf{{#1}}}}
    \newcommand{\NormalTok}[1]{{#1}}
    
    % Additional commands for more recent versions of Pandoc
    \newcommand{\ConstantTok}[1]{\textcolor[rgb]{0.53,0.00,0.00}{{#1}}}
    \newcommand{\SpecialCharTok}[1]{\textcolor[rgb]{0.25,0.44,0.63}{{#1}}}
    \newcommand{\VerbatimStringTok}[1]{\textcolor[rgb]{0.25,0.44,0.63}{{#1}}}
    \newcommand{\SpecialStringTok}[1]{\textcolor[rgb]{0.73,0.40,0.53}{{#1}}}
    \newcommand{\ImportTok}[1]{{#1}}
    \newcommand{\DocumentationTok}[1]{\textcolor[rgb]{0.73,0.13,0.13}{\textit{{#1}}}}
    \newcommand{\AnnotationTok}[1]{\textcolor[rgb]{0.38,0.63,0.69}{\textbf{\textit{{#1}}}}}
    \newcommand{\CommentVarTok}[1]{\textcolor[rgb]{0.38,0.63,0.69}{\textbf{\textit{{#1}}}}}
    \newcommand{\VariableTok}[1]{\textcolor[rgb]{0.10,0.09,0.49}{{#1}}}
    \newcommand{\ControlFlowTok}[1]{\textcolor[rgb]{0.00,0.44,0.13}{\textbf{{#1}}}}
    \newcommand{\OperatorTok}[1]{\textcolor[rgb]{0.40,0.40,0.40}{{#1}}}
    \newcommand{\BuiltInTok}[1]{{#1}}
    \newcommand{\ExtensionTok}[1]{{#1}}
    \newcommand{\PreprocessorTok}[1]{\textcolor[rgb]{0.74,0.48,0.00}{{#1}}}
    \newcommand{\AttributeTok}[1]{\textcolor[rgb]{0.49,0.56,0.16}{{#1}}}
    \newcommand{\InformationTok}[1]{\textcolor[rgb]{0.38,0.63,0.69}{\textbf{\textit{{#1}}}}}
    \newcommand{\WarningTok}[1]{\textcolor[rgb]{0.38,0.63,0.69}{\textbf{\textit{{#1}}}}}
    
    
    % Define a nice break command that doesn't care if a line doesn't already
    % exist.
    \def\br{\hspace*{\fill} \\* }
    % Math Jax compatibility definitions
    \def\gt{>}
    \def\lt{<}
    \let\Oldtex\TeX
    \let\Oldlatex\LaTeX
    \renewcommand{\TeX}{\textrm{\Oldtex}}
    \renewcommand{\LaTeX}{\textrm{\Oldlatex}}
    % Document parameters
    % Document title
    \title{A Computing Data Science Perspective on Gene Splice Site Identification}
    \author{Hui Chong}
    
    
    
    
% Pygments definitions
\makeatletter
\def\PY@reset{\let\PY@it=\relax \let\PY@bf=\relax%
    \let\PY@ul=\relax \let\PY@tc=\relax%
    \let\PY@bc=\relax \let\PY@ff=\relax}
\def\PY@tok#1{\csname PY@tok@#1\endcsname}
\def\PY@toks#1+{\ifx\relax#1\empty\else%
    \PY@tok{#1}\expandafter\PY@toks\fi}
\def\PY@do#1{\PY@bc{\PY@tc{\PY@ul{%
    \PY@it{\PY@bf{\PY@ff{#1}}}}}}}
\def\PY#1#2{\PY@reset\PY@toks#1+\relax+\PY@do{#2}}

\expandafter\def\csname PY@tok@w\endcsname{\def\PY@tc##1{\textcolor[rgb]{0.73,0.73,0.73}{##1}}}
\expandafter\def\csname PY@tok@c\endcsname{\let\PY@it=\textit\def\PY@tc##1{\textcolor[rgb]{0.25,0.50,0.50}{##1}}}
\expandafter\def\csname PY@tok@cp\endcsname{\def\PY@tc##1{\textcolor[rgb]{0.74,0.48,0.00}{##1}}}
\expandafter\def\csname PY@tok@k\endcsname{\let\PY@bf=\textbf\def\PY@tc##1{\textcolor[rgb]{0.00,0.50,0.00}{##1}}}
\expandafter\def\csname PY@tok@kp\endcsname{\def\PY@tc##1{\textcolor[rgb]{0.00,0.50,0.00}{##1}}}
\expandafter\def\csname PY@tok@kt\endcsname{\def\PY@tc##1{\textcolor[rgb]{0.69,0.00,0.25}{##1}}}
\expandafter\def\csname PY@tok@o\endcsname{\def\PY@tc##1{\textcolor[rgb]{0.40,0.40,0.40}{##1}}}
\expandafter\def\csname PY@tok@ow\endcsname{\let\PY@bf=\textbf\def\PY@tc##1{\textcolor[rgb]{0.67,0.13,1.00}{##1}}}
\expandafter\def\csname PY@tok@nb\endcsname{\def\PY@tc##1{\textcolor[rgb]{0.00,0.50,0.00}{##1}}}
\expandafter\def\csname PY@tok@nf\endcsname{\def\PY@tc##1{\textcolor[rgb]{0.00,0.00,1.00}{##1}}}
\expandafter\def\csname PY@tok@nc\endcsname{\let\PY@bf=\textbf\def\PY@tc##1{\textcolor[rgb]{0.00,0.00,1.00}{##1}}}
\expandafter\def\csname PY@tok@nn\endcsname{\let\PY@bf=\textbf\def\PY@tc##1{\textcolor[rgb]{0.00,0.00,1.00}{##1}}}
\expandafter\def\csname PY@tok@ne\endcsname{\let\PY@bf=\textbf\def\PY@tc##1{\textcolor[rgb]{0.82,0.25,0.23}{##1}}}
\expandafter\def\csname PY@tok@nv\endcsname{\def\PY@tc##1{\textcolor[rgb]{0.10,0.09,0.49}{##1}}}
\expandafter\def\csname PY@tok@no\endcsname{\def\PY@tc##1{\textcolor[rgb]{0.53,0.00,0.00}{##1}}}
\expandafter\def\csname PY@tok@nl\endcsname{\def\PY@tc##1{\textcolor[rgb]{0.63,0.63,0.00}{##1}}}
\expandafter\def\csname PY@tok@ni\endcsname{\let\PY@bf=\textbf\def\PY@tc##1{\textcolor[rgb]{0.60,0.60,0.60}{##1}}}
\expandafter\def\csname PY@tok@na\endcsname{\def\PY@tc##1{\textcolor[rgb]{0.49,0.56,0.16}{##1}}}
\expandafter\def\csname PY@tok@nt\endcsname{\let\PY@bf=\textbf\def\PY@tc##1{\textcolor[rgb]{0.00,0.50,0.00}{##1}}}
\expandafter\def\csname PY@tok@nd\endcsname{\def\PY@tc##1{\textcolor[rgb]{0.67,0.13,1.00}{##1}}}
\expandafter\def\csname PY@tok@s\endcsname{\def\PY@tc##1{\textcolor[rgb]{0.73,0.13,0.13}{##1}}}
\expandafter\def\csname PY@tok@sd\endcsname{\let\PY@it=\textit\def\PY@tc##1{\textcolor[rgb]{0.73,0.13,0.13}{##1}}}
\expandafter\def\csname PY@tok@si\endcsname{\let\PY@bf=\textbf\def\PY@tc##1{\textcolor[rgb]{0.73,0.40,0.53}{##1}}}
\expandafter\def\csname PY@tok@se\endcsname{\let\PY@bf=\textbf\def\PY@tc##1{\textcolor[rgb]{0.73,0.40,0.13}{##1}}}
\expandafter\def\csname PY@tok@sr\endcsname{\def\PY@tc##1{\textcolor[rgb]{0.73,0.40,0.53}{##1}}}
\expandafter\def\csname PY@tok@ss\endcsname{\def\PY@tc##1{\textcolor[rgb]{0.10,0.09,0.49}{##1}}}
\expandafter\def\csname PY@tok@sx\endcsname{\def\PY@tc##1{\textcolor[rgb]{0.00,0.50,0.00}{##1}}}
\expandafter\def\csname PY@tok@m\endcsname{\def\PY@tc##1{\textcolor[rgb]{0.40,0.40,0.40}{##1}}}
\expandafter\def\csname PY@tok@gh\endcsname{\let\PY@bf=\textbf\def\PY@tc##1{\textcolor[rgb]{0.00,0.00,0.50}{##1}}}
\expandafter\def\csname PY@tok@gu\endcsname{\let\PY@bf=\textbf\def\PY@tc##1{\textcolor[rgb]{0.50,0.00,0.50}{##1}}}
\expandafter\def\csname PY@tok@gd\endcsname{\def\PY@tc##1{\textcolor[rgb]{0.63,0.00,0.00}{##1}}}
\expandafter\def\csname PY@tok@gi\endcsname{\def\PY@tc##1{\textcolor[rgb]{0.00,0.63,0.00}{##1}}}
\expandafter\def\csname PY@tok@gr\endcsname{\def\PY@tc##1{\textcolor[rgb]{1.00,0.00,0.00}{##1}}}
\expandafter\def\csname PY@tok@ge\endcsname{\let\PY@it=\textit}
\expandafter\def\csname PY@tok@gs\endcsname{\let\PY@bf=\textbf}
\expandafter\def\csname PY@tok@gp\endcsname{\let\PY@bf=\textbf\def\PY@tc##1{\textcolor[rgb]{0.00,0.00,0.50}{##1}}}
\expandafter\def\csname PY@tok@go\endcsname{\def\PY@tc##1{\textcolor[rgb]{0.53,0.53,0.53}{##1}}}
\expandafter\def\csname PY@tok@gt\endcsname{\def\PY@tc##1{\textcolor[rgb]{0.00,0.27,0.87}{##1}}}
\expandafter\def\csname PY@tok@err\endcsname{\def\PY@bc##1{\setlength{\fboxsep}{0pt}\fcolorbox[rgb]{1.00,0.00,0.00}{1,1,1}{\strut ##1}}}
\expandafter\def\csname PY@tok@kc\endcsname{\let\PY@bf=\textbf\def\PY@tc##1{\textcolor[rgb]{0.00,0.50,0.00}{##1}}}
\expandafter\def\csname PY@tok@kd\endcsname{\let\PY@bf=\textbf\def\PY@tc##1{\textcolor[rgb]{0.00,0.50,0.00}{##1}}}
\expandafter\def\csname PY@tok@kn\endcsname{\let\PY@bf=\textbf\def\PY@tc##1{\textcolor[rgb]{0.00,0.50,0.00}{##1}}}
\expandafter\def\csname PY@tok@kr\endcsname{\let\PY@bf=\textbf\def\PY@tc##1{\textcolor[rgb]{0.00,0.50,0.00}{##1}}}
\expandafter\def\csname PY@tok@bp\endcsname{\def\PY@tc##1{\textcolor[rgb]{0.00,0.50,0.00}{##1}}}
\expandafter\def\csname PY@tok@fm\endcsname{\def\PY@tc##1{\textcolor[rgb]{0.00,0.00,1.00}{##1}}}
\expandafter\def\csname PY@tok@vc\endcsname{\def\PY@tc##1{\textcolor[rgb]{0.10,0.09,0.49}{##1}}}
\expandafter\def\csname PY@tok@vg\endcsname{\def\PY@tc##1{\textcolor[rgb]{0.10,0.09,0.49}{##1}}}
\expandafter\def\csname PY@tok@vi\endcsname{\def\PY@tc##1{\textcolor[rgb]{0.10,0.09,0.49}{##1}}}
\expandafter\def\csname PY@tok@vm\endcsname{\def\PY@tc##1{\textcolor[rgb]{0.10,0.09,0.49}{##1}}}
\expandafter\def\csname PY@tok@sa\endcsname{\def\PY@tc##1{\textcolor[rgb]{0.73,0.13,0.13}{##1}}}
\expandafter\def\csname PY@tok@sb\endcsname{\def\PY@tc##1{\textcolor[rgb]{0.73,0.13,0.13}{##1}}}
\expandafter\def\csname PY@tok@sc\endcsname{\def\PY@tc##1{\textcolor[rgb]{0.73,0.13,0.13}{##1}}}
\expandafter\def\csname PY@tok@dl\endcsname{\def\PY@tc##1{\textcolor[rgb]{0.73,0.13,0.13}{##1}}}
\expandafter\def\csname PY@tok@s2\endcsname{\def\PY@tc##1{\textcolor[rgb]{0.73,0.13,0.13}{##1}}}
\expandafter\def\csname PY@tok@sh\endcsname{\def\PY@tc##1{\textcolor[rgb]{0.73,0.13,0.13}{##1}}}
\expandafter\def\csname PY@tok@s1\endcsname{\def\PY@tc##1{\textcolor[rgb]{0.73,0.13,0.13}{##1}}}
\expandafter\def\csname PY@tok@mb\endcsname{\def\PY@tc##1{\textcolor[rgb]{0.40,0.40,0.40}{##1}}}
\expandafter\def\csname PY@tok@mf\endcsname{\def\PY@tc##1{\textcolor[rgb]{0.40,0.40,0.40}{##1}}}
\expandafter\def\csname PY@tok@mh\endcsname{\def\PY@tc##1{\textcolor[rgb]{0.40,0.40,0.40}{##1}}}
\expandafter\def\csname PY@tok@mi\endcsname{\def\PY@tc##1{\textcolor[rgb]{0.40,0.40,0.40}{##1}}}
\expandafter\def\csname PY@tok@il\endcsname{\def\PY@tc##1{\textcolor[rgb]{0.40,0.40,0.40}{##1}}}
\expandafter\def\csname PY@tok@mo\endcsname{\def\PY@tc##1{\textcolor[rgb]{0.40,0.40,0.40}{##1}}}
\expandafter\def\csname PY@tok@ch\endcsname{\let\PY@it=\textit\def\PY@tc##1{\textcolor[rgb]{0.25,0.50,0.50}{##1}}}
\expandafter\def\csname PY@tok@cm\endcsname{\let\PY@it=\textit\def\PY@tc##1{\textcolor[rgb]{0.25,0.50,0.50}{##1}}}
\expandafter\def\csname PY@tok@cpf\endcsname{\let\PY@it=\textit\def\PY@tc##1{\textcolor[rgb]{0.25,0.50,0.50}{##1}}}
\expandafter\def\csname PY@tok@c1\endcsname{\let\PY@it=\textit\def\PY@tc##1{\textcolor[rgb]{0.25,0.50,0.50}{##1}}}
\expandafter\def\csname PY@tok@cs\endcsname{\let\PY@it=\textit\def\PY@tc##1{\textcolor[rgb]{0.25,0.50,0.50}{##1}}}

\def\PYZbs{\char`\\}
\def\PYZus{\char`\_}
\def\PYZob{\char`\{}
\def\PYZcb{\char`\}}
\def\PYZca{\char`\^}
\def\PYZam{\char`\&}
\def\PYZlt{\char`\<}
\def\PYZgt{\char`\>}
\def\PYZsh{\char`\#}
\def\PYZpc{\char`\%}
\def\PYZdl{\char`\$}
\def\PYZhy{\char`\-}
\def\PYZsq{\char`\'}
\def\PYZdq{\char`\"}
\def\PYZti{\char`\~}
% for compatibility with earlier versions
\def\PYZat{@}
\def\PYZlb{[}
\def\PYZrb{]}
\makeatother


    % For linebreaks inside Verbatim environment from package fancyvrb. 
    \makeatletter
        \newbox\Wrappedcontinuationbox 
        \newbox\Wrappedvisiblespacebox 
        \newcommand*\Wrappedvisiblespace {\textcolor{red}{\textvisiblespace}} 
        \newcommand*\Wrappedcontinuationsymbol {\textcolor{red}{\llap{\tiny$\m@th\hookrightarrow$}}} 
        \newcommand*\Wrappedcontinuationindent {3ex } 
        \newcommand*\Wrappedafterbreak {\kern\Wrappedcontinuationindent\copy\Wrappedcontinuationbox} 
        % Take advantage of the already applied Pygments mark-up to insert 
        % potential linebreaks for TeX processing. 
        %        {, <, #, %, $, ' and ": go to next line. 
        %        _, }, ^, &, >, - and ~: stay at end of broken line. 
        % Use of \textquotesingle for straight quote. 
        \newcommand*\Wrappedbreaksatspecials {% 
            \def\PYGZus{\discretionary{\char`\_}{\Wrappedafterbreak}{\char`\_}}% 
            \def\PYGZob{\discretionary{}{\Wrappedafterbreak\char`\{}{\char`\{}}% 
            \def\PYGZcb{\discretionary{\char`\}}{\Wrappedafterbreak}{\char`\}}}% 
            \def\PYGZca{\discretionary{\char`\^}{\Wrappedafterbreak}{\char`\^}}% 
            \def\PYGZam{\discretionary{\char`\&}{\Wrappedafterbreak}{\char`\&}}% 
            \def\PYGZlt{\discretionary{}{\Wrappedafterbreak\char`\<}{\char`\<}}% 
            \def\PYGZgt{\discretionary{\char`\>}{\Wrappedafterbreak}{\char`\>}}% 
            \def\PYGZsh{\discretionary{}{\Wrappedafterbreak\char`\#}{\char`\#}}% 
            \def\PYGZpc{\discretionary{}{\Wrappedafterbreak\char`\%}{\char`\%}}% 
            \def\PYGZdl{\discretionary{}{\Wrappedafterbreak\char`\$}{\char`\$}}% 
            \def\PYGZhy{\discretionary{\char`\-}{\Wrappedafterbreak}{\char`\-}}% 
            \def\PYGZsq{\discretionary{}{\Wrappedafterbreak\textquotesingle}{\textquotesingle}}% 
            \def\PYGZdq{\discretionary{}{\Wrappedafterbreak\char`\"}{\char`\"}}% 
            \def\PYGZti{\discretionary{\char`\~}{\Wrappedafterbreak}{\char`\~}}% 
        } 
        % Some characters . , ; ? ! / are not pygmentized. 
        % This macro makes them "active" and they will insert potential linebreaks 
        \newcommand*\Wrappedbreaksatpunct {% 
            \lccode`\~`\.\lowercase{\def~}{\discretionary{\hbox{\char`\.}}{\Wrappedafterbreak}{\hbox{\char`\.}}}% 
            \lccode`\~`\,\lowercase{\def~}{\discretionary{\hbox{\char`\,}}{\Wrappedafterbreak}{\hbox{\char`\,}}}% 
            \lccode`\~`\;\lowercase{\def~}{\discretionary{\hbox{\char`\;}}{\Wrappedafterbreak}{\hbox{\char`\;}}}% 
            \lccode`\~`\:\lowercase{\def~}{\discretionary{\hbox{\char`\:}}{\Wrappedafterbreak}{\hbox{\char`\:}}}% 
            \lccode`\~`\?\lowercase{\def~}{\discretionary{\hbox{\char`\?}}{\Wrappedafterbreak}{\hbox{\char`\?}}}% 
            \lccode`\~`\!\lowercase{\def~}{\discretionary{\hbox{\char`\!}}{\Wrappedafterbreak}{\hbox{\char`\!}}}% 
            \lccode`\~`\/\lowercase{\def~}{\discretionary{\hbox{\char`\/}}{\Wrappedafterbreak}{\hbox{\char`\/}}}% 
            \catcode`\.\active
            \catcode`\,\active 
            \catcode`\;\active
            \catcode`\:\active
            \catcode`\?\active
            \catcode`\!\active
            \catcode`\/\active 
            \lccode`\~`\~ 	
        }
    \makeatother

    \let\OriginalVerbatim=\Verbatim
    \makeatletter
    \renewcommand{\Verbatim}[1][1]{%
        %\parskip\z@skip
        \sbox\Wrappedcontinuationbox {\Wrappedcontinuationsymbol}%
        \sbox\Wrappedvisiblespacebox {\FV@SetupFont\Wrappedvisiblespace}%
        \def\FancyVerbFormatLine ##1{\hsize\linewidth
            \vtop{\raggedright\hyphenpenalty\z@\exhyphenpenalty\z@
                \doublehyphendemerits\z@\finalhyphendemerits\z@
                \strut ##1\strut}%
        }%
        % If the linebreak is at a space, the latter will be displayed as visible
        % space at end of first line, and a continuation symbol starts next line.
        % Stretch/shrink are however usually zero for typewriter font.
        \def\FV@Space {%
            \nobreak\hskip\z@ plus\fontdimen3\font minus\fontdimen4\font
            \discretionary{\copy\Wrappedvisiblespacebox}{\Wrappedafterbreak}
            {\kern\fontdimen2\font}%
        }%
        
        % Allow breaks at special characters using \PYG... macros.
        \Wrappedbreaksatspecials
        % Breaks at punctuation characters . , ; ? ! and / need catcode=\active 	
        \OriginalVerbatim[#1,codes*=\Wrappedbreaksatpunct]%
    }
    \makeatother

    % Exact colors from NB
    \definecolor{incolor}{HTML}{303F9F}
    \definecolor{outcolor}{HTML}{D84315}
    \definecolor{cellborder}{HTML}{CFCFCF}
    \definecolor{cellbackground}{HTML}{F7F7F7}
    
    % prompt
    \makeatletter
    \newcommand{\boxspacing}{\kern\kvtcb@left@rule\kern\kvtcb@boxsep}
    \makeatother
    \newcommand{\prompt}[4]{
        \ttfamily\llap{{\color{#2}[#3]:\hspace{3pt}#4}}\vspace{-\baselineskip}
    }
    

    
    % Prevent overflowing lines due to hard-to-break entities
    \sloppy 
    % Setup hyperref package
    \hypersetup{
      breaklinks=true,  % so long urls are correctly broken across lines
      colorlinks=true,
      urlcolor=urlcolor,
      linkcolor=linkcolor,
      citecolor=citecolor,
      }
    % Slightly bigger margins than the latex defaults
    
    \geometry{verbose,tmargin=1in,bmargin=1in,lmargin=1in,rmargin=1in}
    
    

\begin{document}
    
    \maketitle
    
    


    \hypertarget{abstract}{%
\section{Abstract}\label{abstract}}

As molecular biology and information technology advances, machine
learning techniques have a wide range of applications in bioinformatics.
In this work, on the problem of gene splicing donor site identification,
on both balanced dataset and unbalanced dataset, three models (WAM
{[}1{]}, BN {[}2,3{]}, SVM {[}4,5{]}) are evaluated comprehensively. A
set of comprehensive performance metrics were also introduced in the
experiments. And to detect splice signals precisely, a correction has
also made to Bayesian network. The result shows that SVM has a good
ability on take caring of unbalanced data while WAM and BN do not.
Besides, the fitness of metrics is also tested. It shows that \(auPRC\)
{[}6{]} is more sensitive to unbalanced data and so is more applicable
in many situations.

    \hypertarget{brief-introduction}{%
\section{Brief Introduction}\label{brief-introduction}}

As molecular biology has an insight into gene regulation and RNA
splicing, recognition of RNA splice site became a key problem. In the
recent 2 decades, several methods have been proposed to solve the
problem. But their ability of take care of unbalanced data have not been
tested yet. In this work, a test of WAM, BN, SVM on both balanced data
and unbalanced data were performed.

    \hypertarget{splice-site}{%
\subsection{Splice Site}\label{splice-site}}

Splice site {[}7{]} , in molecular biology, is a position on RNA where
RNA splicing occurs. RNA splicing is a form of RNA processing in which a
newly made precursor messenger RNA (pre-mRNA) transcript is transformed
into a mature messenger RNA (mRNA). After splicing, introns (Non-coding
regions) are removed and exons (Coding Regions) are joined together.

There are donor sites (5' end) and acceptor sites (3' end) within
introns {[}28{]}.

    \hypertarget{models}{%
\subsection{Models}\label{models}}

As Biology advances, computational methods are now the only realistic
way to answer many questions in modern biology {[}29{]}. For the problem
of splice site recognition, there are several models have been peoposed.

\textbf{Weighted Array Model} {[}1{]} is a frequency-based approach. It
calculate the frequency of each adjacent bases in specific positions to
fill into a weight array matrix, and use weight matrix to score unknown
site. It has the ability to detect some signals like common bases amoung
donors. The limitation of this approach, in a machine learning
perspective, is its poor ability to automatically calculate hidden
features, like GC content. There might be much more complex features
that contribute to splice signal transduction.

\textbf{Bayesian Network} {[}2,3{]} is a Bayesian method. It learns DAG
structure and conditional probabilities distribution from the data. It
has the ability to take care of dependence between positions. It also
has the ability to imputate the missing value of the data. and besides,
it can work like a Bayesian classifier when there is conditional
probability from features to label of each sample. The limitation of
this approach, is its unprecise and ineffective structure learning
algorithm. PC-Constraints based learning algorithm forms loops in DAG
and causes errors sometimes. MMHC algorithm is too slow when there is
hundreds or thousands of nodes. Structrual expectation maximization is
quite good, but still needs prior expert knowledges.

\textbf{Support Vector Machine} {[}4,5{]} is a mature supervised
learning method. It employs support vectors to maximize the margin and
kernel trick to classify linearly indivisible data. This allows for
significant savings in calculations. But detecting signals via SVM is
not a good choice, because over three-dimensional hyper-plane within SVM
cannot be visualized and analyzed. This is the flaw of SVM in data
mining, it has the ability to classify but is almost useless for human
interpretation of results.

    \hypertarget{material-and-method}{%
\section{Material and Methods}\label{material-and-method}}

    \hypertarget{environment}{%
\subsection{Environment}\label{environment}}

    \begin{tcolorbox}[breakable, size=fbox, boxrule=1pt, pad at break*=1mm,colback=cellbackground, colframe=cellborder]
\prompt{In}{incolor}{27}{\boxspacing}
\begin{Verbatim}[commandchars=\\\{\},fontsize=\small]
\PY{k+kn}{import} \PY{n+nn}{pickle}
\PY{k+kn}{import} \PY{n+nn}{warnings}
\PY{k+kn}{import} \PY{n+nn}{importlib}
\PY{k+kn}{import} \PY{n+nn}{import\PYZus{}ipynb}
\PY{k+kn}{import} \PY{n+nn}{numpy} \PY{k}{as} \PY{n+nn}{np}
\PY{k+kn}{import} \PY{n+nn}{pandas} \PY{k}{as} \PY{n+nn}{pd}
\PY{k+kn}{import} \PY{n+nn}{seaborn} \PY{k}{as} \PY{n+nn}{sns}
\PY{k+kn}{import} \PY{n+nn}{networkx} \PY{k}{as} \PY{n+nn}{nx}
\PY{k+kn}{import} \PY{n+nn}{matplotlib}\PY{n+nn}{.}\PY{n+nn}{image} \PY{k}{as} \PY{n+nn}{pli}
\PY{k+kn}{from} \PY{n+nn}{Models} \PY{k+kn}{import} \PY{n}{WAM}\PY{p}{,} \PY{n}{BN}\PY{p}{,} \PY{n}{SVM}
\PY{k+kn}{from} \PY{n+nn}{Evaluator} \PY{k+kn}{import} \PY{n}{Evaluator}
\PY{k+kn}{from} \PY{n+nn}{matplotlib} \PY{k+kn}{import} \PY{n}{pyplot} \PY{k}{as} \PY{n}{plt}
\PY{k+kn}{from} \PY{n+nn}{Utils} \PY{k+kn}{import} \PY{n}{load\PYZus{}data}\PY{p}{,} \PY{n}{Summary}\PY{p}{,} \PY{n}{Validation\PYZus{}Sum}\PY{p}{,} \PY{n}{no\PYZus{}box}
\PY{k+kn}{from} \PY{n+nn}{IPython}\PY{n+nn}{.}\PY{n+nn}{display} \PY{k+kn}{import} \PY{n}{display}\PY{p}{,} \PY{n}{Markdown}\PY{p}{,} \PY{n}{Latex}\PY{p}{,} \PY{n}{HTML}
\PY{k+kn}{from} \PY{n+nn}{matplotlib}\PY{n+nn}{.}\PY{n+nn}{patches} \PY{k+kn}{import} \PY{n}{ConnectionPatch}
\PY{k+kn}{from} \PY{n+nn}{mpl\PYZus{}toolkits}\PY{n+nn}{.}\PY{n+nn}{axes\PYZus{}grid1}\PY{n+nn}{.}\PY{n+nn}{inset\PYZus{}locator} \PY{k+kn}{import} \PY{n}{mark\PYZus{}inset}
\PY{k+kn}{from} \PY{n+nn}{mpl\PYZus{}toolkits}\PY{n+nn}{.}\PY{n+nn}{axes\PYZus{}grid1}\PY{n+nn}{.}\PY{n+nn}{inset\PYZus{}locator} \PY{k+kn}{import} \PY{n}{inset\PYZus{}axes}
\PY{k+kn}{from} \PY{n+nn}{pgmpy}\PY{n+nn}{.}\PY{n+nn}{estimators} \PY{k+kn}{import} \PY{n}{MmhcEstimator}\PY{p}{,} \PY{n}{ConstraintBasedEstimator}

\PY{n}{plt}\PY{o}{.}\PY{n}{ion}\PY{p}{(}\PY{p}{)}
\PY{n}{plt}\PY{o}{.}\PY{n}{style}\PY{o}{.}\PY{n}{use}\PY{p}{(}\PY{p}{[}\PY{l+s+s1}{\PYZsq{}}\PY{l+s+s1}{science}\PY{l+s+s1}{\PYZsq{}}\PY{p}{,}\PY{l+s+s1}{\PYZsq{}}\PY{l+s+s1}{ieee}\PY{l+s+s1}{\PYZsq{}}\PY{p}{,}\PY{l+s+s1}{\PYZsq{}}\PY{l+s+s1}{no\PYZhy{}latex}\PY{l+s+s1}{\PYZsq{}}\PY{p}{]}\PY{p}{)}
\PY{n}{warnings}\PY{o}{.}\PY{n}{filterwarnings}\PY{p}{(}\PY{l+s+s2}{\PYZdq{}}\PY{l+s+s2}{ignore}\PY{l+s+s2}{\PYZdq{}}\PY{p}{)}
\end{Verbatim}
\end{tcolorbox}

    \begin{Verbatim}[commandchars=\\\{\},fontsize=\small]
importing Jupyter notebook from Utils.ipynb
    \end{Verbatim}

    \hypertarget{data-sets}{%
\subsection{Datasets}\label{data-sets}}

To comprehensively compare their performances, two datasets were used
as data are usually unbalanced in Biology. Models may perform
differently in that way. The first dataset, which is balanced, was used
to training and testing models' performance in a general way. And the
second dataset is an unbalanced dataset, was used to training and thus
testing their performance like real conditions. Because unbalanced data
is common in Biology and even many other fields.

Basic information of the two datasets is as follows.

    \begin{tcolorbox}[breakable, size=fbox, boxrule=1pt, pad at break*=1mm,colback=cellbackground, colframe=cellborder]
\prompt{In}{incolor}{28}{\boxspacing}
\begin{Verbatim}[commandchars=\\\{\},fontsize=\small]
\PY{n}{training\PYZus{}set}\PY{p}{,} \PY{n}{testing\PYZus{}set} \PY{o}{=} \PY{n}{load\PYZus{}data}\PY{p}{(}\PY{p}{)}
\PY{n}{training\PYZus{}set1} \PY{o}{=} \PY{n}{training\PYZus{}set}\PY{p}{[}\PY{l+m+mi}{0}\PY{p}{:}\PY{l+m+mi}{5000}\PY{p}{]}
\PY{n}{testing\PYZus{}set1} \PY{o}{=} \PY{n}{testing\PYZus{}set}\PY{p}{[}\PY{l+m+mi}{0}\PY{p}{:}\PY{l+m+mi}{5000}\PY{p}{]}
\PY{n}{training\PYZus{}set2} \PY{o}{=} \PY{n}{training\PYZus{}set}\PY{p}{[}\PY{l+m+mi}{1500}\PY{p}{:}\PY{l+m+mi}{6500}\PY{p}{]}\PY{o}{.}\PY{n}{reset\PYZus{}index}\PY{p}{(}\PY{n}{drop}\PY{o}{=}\PY{k+kc}{True}\PY{p}{)}
\PY{n}{testing\PYZus{}set2} \PY{o}{=} \PY{n}{testing\PYZus{}set}\PY{p}{[}\PY{l+m+mi}{1500}\PY{p}{:}\PY{l+m+mi}{6500}\PY{p}{]}\PY{o}{.}\PY{n}{reset\PYZus{}index}\PY{p}{(}\PY{n}{drop}\PY{o}{=}\PY{k+kc}{True}\PY{p}{)}
\PY{n+nb}{print}\PY{p}{(}\PY{n}{Summary}\PY{p}{(}\PY{n}{training\PYZus{}set1}\PY{p}{,} \PY{n}{testing\PYZus{}set1}\PY{p}{,} \PY{n}{training\PYZus{}set2}\PY{p}{,} \PY{n}{testing\PYZus{}set2}\PY{p}{)}\PY{o}{.}\PY{n}{to\PYZus{}latex}\PY{p}{(}\PY{p}{)}\PY{p}{)}
\end{Verbatim}
\end{tcolorbox}

\begin{table*}[h] %\label{table:number}
\centering
\caption{Basic information of datasets} 
\begin{threeparttable}
\begin{tabular}{cccl}
	\toprule
	{} &  nSample$^a$ & nDonor : nPseudo$^b$ &        Type \\
	\midrule
	Dataset 1 &    10000 &      5278 : 4722 &    Balanced \\
	Dataset 2 &    10000 &      2278 : 7722 &  Unbalanced \\
	\bottomrule
\end{tabular}
 \begin{tablenotes}
	\footnotesize
	\item[a] Number of samples in the dataset.
	\item[b] Proportion of donor sites and pseudo sites in the dataset.
	%\item[c] Type of the dataset.
\end{tablenotes}
\end{threeparttable}
\end{table*}




    \hypertarget{feature-extraction-and-encoding}{%
\subsection{Feature Extraction and
Encoding}\label{feature-extraction-and-encoding}}

To build reliable models and test their performances, high-quanlity data
must be extracted. Splice site dataset, which contains \(5278\) donor
sites and \(4722\) pseudo sites, was extracted from original sequence
data. And only \(10\) bases before and after the site to be tested were
extracted.

When encoding sequence features, one-hot encoder {[}8{]} was introduced.
And considering simplisity of the experiments, samples with
\(base_{ambiguous} \in \{w,m,s,k,r,y,b,d,h,v,n,z\}\) {[}9{]} were
discarded. Next, a One-hot encoder are used to encode sequence to a
numeric feature vector. For instance, sequence \(a...t\) was encoded as
\([1,0,0,0...0,0,0,1]\). See
\href{https://github.com/AdeBC/GSSR/blob/master/Source/Preprocessing.ipynb}{Preprocessing.ipynb}
for details.

    \hypertarget{model-implements-and-test-core}{%
\subsection{Model Implements and Test
score}\label{model-implements-and-test-score}}

The 3 models are mainly implemented using Python {[}10{]}. Besides, the
Bayesian network model and Support vector machine was implemented based
on Pgmpy and Scikit-learn packages {[}11,12{]}. As far as possible,
these models have been designed to be practically applicable. They
contain several very easy-to-use program interfaces: \texttt{fit},
\texttt{predict\_probas}, \texttt{predict\_scores} and
\texttt{predict\_classes}. For all 3 models, the \texttt{fit} method is
used for training, and the other 3 \texttt{predict} methods are used for
prediction. The \texttt{predict\_probas} is used for predicting the
probabilities each sample belonging to each class, the
\texttt{predict\_scores} are used for calculating the \(S(X)\), and the
\texttt{predict\_classes} are used for predicting the class to which
each sample belongs, this is done by comparing the \(S(X)\) and given
threshold \(T\). Most importantly, the \(S(X)\) is calculated by:
\[S_m(X) = ln(\frac{P_m^+ (X)}{P_m^- (X)}), m\in\{wam, bn, svm\} \tag{1}\]

    \hypertarget{stratagies-for-wam-to-perform-recognition}{%
\subsection{Stratagies for WAM to perform
recognition}\label{stratagies-for-wam-to-perform-recognition}}

When recoginizing Gene splice site, the 3 steps were mainly used to
determine the potential of each sample being a splice site. Firstly,
Laplace pseudocount \(C_p\) {[}13{]} are used to smooth categorical
data. Secondly, the weights of positive and negative submodels, which
are the frequencies of adjacent base pairs in WAM, are calculated using
the following formula.
\[f(x) = \frac{C_{x,p}+C_p}{N},x \in \{aa,ac,ag,...tt\} \tag{2} \] Where
\(N\) is the total number of samples. \(C_{x,p}\) is the total number of
specific adjacent base pairs occurring at position \(p-l+1...p\) in all
samples. \(C_p\) is the pseudocount, and \(l\) is the length of array
considered, which refers to the range of dependence between adjacent
base pairs considered in experiments. And finally, the potential of
unknown samples being donor sites were given by
\[S_{wam}(X) = ln(\frac{P^+(X)}{P^-(X)}) = \sum_{x=2}^{L}{ln(\frac{f^+(x)}{f^-(x)})} \tag{3}\]
Where \(L\) is the length of input sequence. Most importantly, \(l\) and
\(C_p\) are set to \(2\) and \(0.5\) respectively.

    \hypertarget{stratagies-for-bn-to-learn-structures-and-parameters-from-data}{%
\subsection{Stratagies for BN to learn structures and parameters from
data}\label{stratagies-for-bn-to-learn-structures-and-parameters-from-data}}

To perform accurate structure and parameter learning for Bayesian
network, Two algorithms were used: PC constraint-based learning
algorithm {[}14{]} for structure learning and Bayesian estimation for
parameters learning.

The PC algorithm learns the DAG structure in three steps {[}30{]}.
First, find the graph skeleton and separating sets of removed edges.
Second, orient v-structures based on separating sets. Finally, propagate
orientations of v-structures to as many remaining undirected edges as
possible. The Bayesian estimator then estimates conditional probability
distribution for each node in the graph.

    \hypertarget{strategies-for-bn-to-detect-splice-signal}{%
\subsection{Strategies for BN to detect splice
signal}\label{strategies-for-bn-to-detect-splice-signal}}

Because of Bayesian inference, Bayesian network can handle the causalty
between nodes. In this project, Bayesian network have been improved to
be able to calculate the splicing donor potential of DNA sequences. In
which both two kinds of data are retained: the feature nodes
\(N_{f}, f\in \{-10,-9,...10\}\) and label node \(N_L\). In the Bayesian
network, each feature node has \(state_f \in \{a,c,g,t\}\) and label
node has \(state_L\in \{0,1\}\). To enable Bayesian network to learn
conditional probability for all nodes
\(N \in \{N_{-10},N_{-9},...N_{10},N_{L}\}\), both donor sites and
pseudo sites are fitted in a Bayesian network. Finally, when predicting
new sites, the Bayesian network will treat new sites as sample with
missing label. It uses a missing data imputation algorithm to calculate
the probability of label in each state, which then gives the donor
potential. See
\href{https://github.com/AdeBC/GSSR/blob/master/Source/Models.ipynb}{Models.ipynb}
for implements and 2.3 test score for donor potential calculation.

    \hypertarget{correction}{%
\subsection{Correction}\label{correction}}

Since Bayesian networks cannot perfectly infer causality between \(N_f\)
and \(N_L\), correction to directions of edges was performed. After
construting a DAG structure, the edges from \(N_L\) to
\(N_f, f \in \{−10,−9,...10\}\) are reversed (see
\href{https://github.com/AdeBC/GSSR/blob/master/Source/Models.ipynb}{Models.ipynb}).
For example, the edge \(N_L \rightarrow N_{-1}\) are reversed into
\(N_L \leftarrow N_{-1}\).

    \hypertarget{settings-for-svm-to-classify-splice-sites}{%
\subsection{Settings for SVM to classify splice
sites}\label{settings-for-svm-to-classify-splice-sites}}

In the experiment, Gaussian Kernel function {[}15{]} were used to map
vectors to higher-dimensional feature space. The \(\gamma\) for Gaussian
Kernel are set to \$ 1/(Num\_\{features\}Variance\_\{data\})\$. All
other parameters are default values (Scikit-learn version 0.23.1). See
details in
\href{https://scikit-learn.org/stable/modules/generated/sklearn.svm.SVC.html\#sklearn.svm.SVC}{Sklearn
documentation}.

\[K(x_i, x_j) = exp(-\gamma||x_i - x_j||^2), \gamma > 0\ \tag{4}\]


    \begin{tcolorbox}[breakable, size=fbox, boxrule=1pt, pad at break*=1mm,colback=cellbackground, colframe=cellborder]
\prompt{In}{incolor}{3}{\boxspacing}
\begin{Verbatim}[commandchars=\\\{\},fontsize=\small]
\PY{k}{class} \PY{n+nc}{Model}\PY{p}{:}
    
    \PY{k}{def} \PY{n+nf+fm}{\PYZus{}\PYZus{}init\PYZus{}\PYZus{}}\PY{p}{(}\PY{n+nb+bp}{self}\PY{p}{)}\PY{p}{:}
        \PY{n+nb+bp}{self}\PY{o}{.}\PY{n}{wam} \PY{o}{=} \PY{n}{WAM}\PY{p}{(}\PY{p}{)}
        \PY{n+nb+bp}{self}\PY{o}{.}\PY{n}{bn} \PY{o}{=} \PY{n}{BN}\PY{p}{(}\PY{n}{struc\PYZus{}estr}\PY{o}{=}\PY{n}{ConstraintBasedEstimator}\PY{p}{)}
        \PY{n+nb+bp}{self}\PY{o}{.}\PY{n}{svm} \PY{o}{=} \PY{n}{SVM}\PY{p}{(}\PY{n}{kernel}\PY{o}{=}\PY{l+s+s1}{\PYZsq{}}\PY{l+s+s1}{rbf}\PY{l+s+s1}{\PYZsq{}}\PY{p}{,} \PY{n}{gamma}\PY{o}{=}\PY{l+s+s1}{\PYZsq{}}\PY{l+s+s1}{scale}\PY{l+s+s1}{\PYZsq{}}\PY{p}{,} \PY{n}{random\PYZus{}state}\PY{o}{=}\PY{l+m+mi}{1}\PY{p}{,} \PY{n}{probability}\PY{o}{=}\PY{k+kc}{True}\PY{p}{,} \PY{n}{verbose}\PY{o}{=}\PY{k+kc}{True}\PY{p}{)}
        
\PY{n}{model\PYZus{}1} \PY{o}{=} \PY{n}{Model}\PY{p}{(}\PY{p}{)}
\PY{n}{model\PYZus{}2} \PY{o}{=} \PY{n}{Model}\PY{p}{(}\PY{p}{)}
\end{Verbatim}
\end{tcolorbox}

    \hypertarget{cross-validation}{%
\subsection{Cross-validation}\label{cross-validation}}

To test the accuracy of three models, a simple cross-validation {[}16{]}
was performed. In both two experiments, independent and identically
distributed testing set with was selected.

    \hypertarget{evaluation}{%
\subsection{Evaluation}\label{evaluation}}

To visualize and compare classification results of 3 models, ROC curves
{[}17{]} and PR curves {[}18{]} are plotted using Matplotlib package
{[}26,27{]}. To realize this, a sample-based evaluation method is
introduced:

    \[ TP_t = \sum_{i=0}^N I(S_i\geq t \land L_i = 1) \tag{5} \]

\[ FP_t = \sum_{i=0}^N I(S_i\geq t \land L_i = 0) \tag{6}  \]

\[ TN_t = \sum_{i=0}^N I(S_i< t \land L_i = 0) \tag{7}  \]

\[ FN_t = \sum_{i=0}^N I(S_i< t \land L_i = 1) \tag{8}  \]

where \(TP_t\), \(FP_t\), \(TP_t\), \(TN_t\) are statistics for models'
prediction result in specific threshold \(t\). \(I(True) = 0\) and
\(I(False) = 1\), \(L_i\) is label for \(ith\) sample. \(S_i\) is the
score calculated based on probability for \(ith\) sample.
\(t\in \{-25, -24.5, -24, ...24.5, 25\}\) is the threshold used in the
evaluation.

And following metrics were used in the evaluation.

\[ TPR_t = \frac{TP_t}{TP_t+FN_t} \tag{9}  \]

\[ FPR_t = \frac{FP_t}{FP_t+TN_t} \tag{10}  \]

\[ Sn_t = \frac{TP_t}{TP_t+FN_t} \tag{11}  \]

\[ Sp_t = \frac{TN_t}{TN_t+FP_t} \tag{12}  \]

\[ Pr_t = \frac{TP_t}{TP_t+FP_t} \tag{13}  \]

\[ Rc_t = \frac{TP_t}{TP_t+FN_t} \tag{14}  \]

\[ F1-score_t = \frac{2Pr_t Rc_t}{Pr_t+Rc_t} \tag{15} \]

    Finally, \(auROC\) (the area under ROC curve) {[}19{]} and \(auPRC\)
(the area under Precision-Recall curve) {[}6,18{]} were calculated using
Composite Trapezoidal rule {[}20{]}:

\[ auROC = \sum_{i=1}^{N-1} \frac{(FPR_{i+1} - FPR_i)(TPR_i+TPR_{i+1})}{2} \tag{16} \]

\[ auPRC = \sum_{i=1}^{N-1} \frac{(Rc_{i+1} - Rc_i)(Pr_i+Pr_{i+1})}{2} \tag{17} \]

For implements, see
\href{https://github.com/AdeBC/GSSR/blob/master/Source/Evaluator.ipynb}{Evaluator.ipynb}.

    \hypertarget{result-and-discussion}{%
\section{Results and Discussion}\label{results-and-discussion}}

In this section we discuss the result of two experiments.

    \hypertarget{model-training}{%
\subsection{Model Training}\label{model-training}}

Three models were trained here. WAM and BN were trained and tested using
sequence features, and SVM was trained and tested using numerical
features. The numerical features were encoded using one-hot encoder (see
2.3 feature extraction).

    \begin{tcolorbox}[breakable, size=fbox, boxrule=1pt, pad at break*=1mm,colback=cellbackground, colframe=cellborder]
\prompt{In}{incolor}{ }{\boxspacing}
\begin{Verbatim}[commandchars=\\\{\},fontsize=\small]
\PY{n}{model\PYZus{}1}\PY{o}{.}\PY{n}{wam}\PY{o}{.}\PY{n}{fit}\PY{p}{(}\PY{n}{training\PYZus{}set1}\PY{p}{[}\PY{l+s+s1}{\PYZsq{}}\PY{l+s+s1}{Seq. features}\PY{l+s+s1}{\PYZsq{}}\PY{p}{]}\PY{p}{,} \PY{n}{training\PYZus{}set1}\PY{p}{[}\PY{l+s+s1}{\PYZsq{}}\PY{l+s+s1}{IsDonor}\PY{l+s+s1}{\PYZsq{}}\PY{p}{]}\PY{p}{)}
\PY{n}{model\PYZus{}2}\PY{o}{.}\PY{n}{wam}\PY{o}{.}\PY{n}{fit}\PY{p}{(}\PY{n}{training\PYZus{}set2}\PY{p}{[}\PY{l+s+s1}{\PYZsq{}}\PY{l+s+s1}{Seq. features}\PY{l+s+s1}{\PYZsq{}}\PY{p}{]}\PY{p}{,} \PY{n}{training\PYZus{}set2}\PY{p}{[}\PY{l+s+s1}{\PYZsq{}}\PY{l+s+s1}{IsDonor}\PY{l+s+s1}{\PYZsq{}}\PY{p}{]}\PY{p}{)}
\end{Verbatim}
\end{tcolorbox}

    \begin{tcolorbox}[breakable, size=fbox, boxrule=1pt, pad at break*=1mm,colback=cellbackground, colframe=cellborder]
\prompt{In}{incolor}{ }{\boxspacing}
\begin{Verbatim}[commandchars=\\\{\},fontsize=\small]
\PY{n}{model\PYZus{}2}\PY{o}{.}\PY{n}{bn}\PY{o}{.}\PY{n}{fit}\PY{p}{(}\PY{n}{seqs}\PY{o}{=}\PY{n}{training\PYZus{}set2}\PY{p}{[}\PY{l+s+s1}{\PYZsq{}}\PY{l+s+s1}{Seq. features}\PY{l+s+s1}{\PYZsq{}}\PY{p}{]}\PY{p}{,} 
               \PY{n}{label}\PY{o}{=}\PY{n}{training\PYZus{}set2}\PY{p}{[}\PY{l+s+s1}{\PYZsq{}}\PY{l+s+s1}{IsDonor}\PY{l+s+s1}{\PYZsq{}}\PY{p}{]}\PY{o}{.}\PY{n}{astype}\PY{p}{(}\PY{n+nb}{int}\PY{p}{)}\PY{p}{,} \PY{n}{sign\PYZus{}level}\PY{o}{=}\PY{l+m+mf}{0.01}\PY{p}{)}
\PY{n}{model\PYZus{}1}\PY{o}{.}\PY{n}{bn}\PY{o}{.}\PY{n}{fit}\PY{p}{(}\PY{n}{seqs}\PY{o}{=}\PY{n}{training\PYZus{}set1}\PY{p}{[}\PY{l+s+s1}{\PYZsq{}}\PY{l+s+s1}{Seq. features}\PY{l+s+s1}{\PYZsq{}}\PY{p}{]}\PY{p}{,} 
               \PY{n}{label}\PY{o}{=}\PY{n}{training\PYZus{}set1}\PY{p}{[}\PY{l+s+s1}{\PYZsq{}}\PY{l+s+s1}{IsDonor}\PY{l+s+s1}{\PYZsq{}}\PY{p}{]}\PY{o}{.}\PY{n}{astype}\PY{p}{(}\PY{n+nb}{int}\PY{p}{)}\PY{p}{,} \PY{n}{sign\PYZus{}level}\PY{o}{=}\PY{l+m+mf}{0.01}\PY{p}{)}
\end{Verbatim}
\end{tcolorbox}

    \begin{tcolorbox}[breakable, size=fbox, boxrule=1pt, pad at break*=1mm,colback=cellbackground, colframe=cellborder]
\prompt{In}{incolor}{ }{\boxspacing}
\begin{Verbatim}[commandchars=\\\{\},fontsize=\small]
\PY{n}{to\PYZus{}format\PYZus{}arr} \PY{o}{=} \PY{k}{lambda} \PY{n}{data}\PY{p}{:} \PY{n}{np}\PY{o}{.}\PY{n}{array}\PY{p}{(}\PY{p}{[}\PY{n}{np}\PY{o}{.}\PY{n}{array}\PY{p}{(}\PY{n+nb}{list}\PY{p}{(}\PY{n}{i}\PY{p}{)}\PY{p}{)} \PY{k}{for} \PY{n}{i} \PY{o+ow}{in} \PY{n}{data}\PY{p}{]}\PY{p}{)}
\PY{n}{model\PYZus{}1}\PY{o}{.}\PY{n}{svm}\PY{o}{.}\PY{n}{fit}\PY{p}{(}\PY{n}{to\PYZus{}format\PYZus{}arr}\PY{p}{(}\PY{n}{training\PYZus{}set1}\PY{p}{[}\PY{l+s+s1}{\PYZsq{}}\PY{l+s+s1}{Num. features}\PY{l+s+s1}{\PYZsq{}}\PY{p}{]}\PY{p}{)}\PY{p}{,} 
                \PY{n}{training\PYZus{}set1}\PY{p}{[}\PY{l+s+s1}{\PYZsq{}}\PY{l+s+s1}{IsDonor}\PY{l+s+s1}{\PYZsq{}}\PY{p}{]}\PY{p}{)}
\PY{n}{model\PYZus{}2}\PY{o}{.}\PY{n}{svm}\PY{o}{.}\PY{n}{fit}\PY{p}{(}\PY{n}{to\PYZus{}format\PYZus{}arr}\PY{p}{(}\PY{n}{training\PYZus{}set2}\PY{p}{[}\PY{l+s+s1}{\PYZsq{}}\PY{l+s+s1}{Num. features}\PY{l+s+s1}{\PYZsq{}}\PY{p}{]}\PY{p}{)}\PY{p}{,} 
                \PY{n}{training\PYZus{}set2}\PY{p}{[}\PY{l+s+s1}{\PYZsq{}}\PY{l+s+s1}{IsDonor}\PY{l+s+s1}{\PYZsq{}}\PY{p}{]}\PY{p}{)}
\end{Verbatim}
\end{tcolorbox}

    \hypertarget{test-scores-calculation}{%
\subsection{Test scores Calculation}\label{test-scores-calculation}}

The test scores for models were calculated here.

    \begin{tcolorbox}[breakable, size=fbox, boxrule=1pt, pad at break*=1mm,colback=cellbackground, colframe=cellborder]
\prompt{In}{incolor}{8}{\boxspacing}
\begin{Verbatim}[commandchars=\\\{\},fontsize=\small]
\PY{n}{scores\PYZus{}wam\PYZus{}1} \PY{o}{=} \PY{n}{model\PYZus{}1}\PY{o}{.}\PY{n}{wam}\PY{o}{.}\PY{n}{predict\PYZus{}scores}\PY{p}{(}\PY{n}{testing\PYZus{}set1}\PY{p}{[}\PY{l+s+s1}{\PYZsq{}}\PY{l+s+s1}{Seq. features}\PY{l+s+s1}{\PYZsq{}}\PY{p}{]}\PY{o}{.}\PY{n}{apply}\PY{p}{(}\PY{n+nb}{str}\PY{o}{.}\PY{n}{upper}\PY{p}{)}\PY{p}{)}
\PY{n}{scores\PYZus{}wam\PYZus{}2} \PY{o}{=} \PY{n}{model\PYZus{}2}\PY{o}{.}\PY{n}{wam}\PY{o}{.}\PY{n}{predict\PYZus{}scores}\PY{p}{(}\PY{n}{testing\PYZus{}set2}\PY{p}{[}\PY{l+s+s1}{\PYZsq{}}\PY{l+s+s1}{Seq. features}\PY{l+s+s1}{\PYZsq{}}\PY{p}{]}\PY{o}{.}\PY{n}{apply}\PY{p}{(}\PY{n+nb}{str}\PY{o}{.}\PY{n}{upper}\PY{p}{)}\PY{p}{)}
\PY{n}{scores\PYZus{}bn\PYZus{}1} \PY{o}{=} \PY{n}{model\PYZus{}1}\PY{o}{.}\PY{n}{bn}\PY{o}{.}\PY{n}{predict\PYZus{}scores}\PY{p}{(}\PY{n}{testing\PYZus{}set1}\PY{p}{[}\PY{l+s+s1}{\PYZsq{}}\PY{l+s+s1}{Seq. features}\PY{l+s+s1}{\PYZsq{}}\PY{p}{]}\PY{p}{)}
\PY{n}{scores\PYZus{}bn\PYZus{}2} \PY{o}{=} \PY{n}{model\PYZus{}2}\PY{o}{.}\PY{n}{bn}\PY{o}{.}\PY{n}{predict\PYZus{}scores}\PY{p}{(}\PY{n}{testing\PYZus{}set2}\PY{p}{[}\PY{l+s+s1}{\PYZsq{}}\PY{l+s+s1}{Seq. features}\PY{l+s+s1}{\PYZsq{}}\PY{p}{]}\PY{p}{)}
\PY{n}{to\PYZus{}format\PYZus{}arr} \PY{o}{=} \PY{k}{lambda} \PY{n}{data}\PY{p}{:} \PY{n}{np}\PY{o}{.}\PY{n}{array}\PY{p}{(}\PY{p}{[}\PY{n}{np}\PY{o}{.}\PY{n}{array}\PY{p}{(}\PY{n+nb}{list}\PY{p}{(}\PY{n}{i}\PY{p}{)}\PY{p}{)} \PY{k}{for} \PY{n}{i} \PY{o+ow}{in} \PY{n}{data}\PY{p}{]}\PY{p}{)}
\PY{n}{scores\PYZus{}svm\PYZus{}1} \PY{o}{=} \PY{n}{model\PYZus{}1}\PY{o}{.}\PY{n}{svm}\PY{o}{.}\PY{n}{predict\PYZus{}scores}\PY{p}{(}\PY{n}{to\PYZus{}format\PYZus{}arr}\PY{p}{(}\PY{n}{testing\PYZus{}set1}\PY{p}{[}\PY{l+s+s1}{\PYZsq{}}\PY{l+s+s1}{Num. features}\PY{l+s+s1}{\PYZsq{}}\PY{p}{]}\PY{p}{)}\PY{p}{)}
\PY{n}{scores\PYZus{}svm\PYZus{}2} \PY{o}{=} \PY{n}{model\PYZus{}2}\PY{o}{.}\PY{n}{svm}\PY{o}{.}\PY{n}{predict\PYZus{}scores}\PY{p}{(}\PY{n}{to\PYZus{}format\PYZus{}arr}\PY{p}{(}\PY{n}{testing\PYZus{}set2}\PY{p}{[}\PY{l+s+s1}{\PYZsq{}}\PY{l+s+s1}{Num. features}\PY{l+s+s1}{\PYZsq{}}\PY{p}{]}\PY{p}{)}\PY{p}{)}
\end{Verbatim}
\end{tcolorbox}

    \hypertarget{result-on-two-dataset}{%
\subsection{Result on two datasets}\label{result-on-two-datasets}}

Here all results of three models in both two experiments were compared.
Above all, The results show that all three models detected splice signal
in a way. Their \(auROC\) are very close to 1, which means they can
perfectly classify donor sites from Pseudo sites. But in contrast, Both
auROC and auPRC showed that SVM has a higher accuracy than the other
(Fig. 1). Moreover, SVM are more robust when trained and tested on
unbalanced dataset than BN and WAM. The \(auPRC\) can show performance
decline significantly while \(auROC\) cannot when models were trained on
unbalanced data.

    \begin{tcolorbox}[breakable, size=fbox, boxrule=1pt, pad at break*=1mm,colback=cellbackground, colframe=cellborder]
\prompt{In}{incolor}{10}{\boxspacing}
\begin{Verbatim}[commandchars=\\\{\},fontsize=\small]
\PY{n}{trange} \PY{o}{=} \PY{n}{np}\PY{o}{.}\PY{n}{arange}\PY{p}{(}\PY{o}{\PYZhy{}}\PY{l+m+mi}{40}\PY{p}{,} \PY{l+m+mi}{40}\PY{p}{,} \PY{l+m+mf}{0.5}\PY{p}{)}
\PY{n}{evaltr\PYZus{}1} \PY{o}{=} \PY{n}{Evaluator}\PY{p}{(}\PY{n}{labels}\PY{o}{=}\PY{n}{testing\PYZus{}set1}\PY{p}{[}\PY{l+s+s1}{\PYZsq{}}\PY{l+s+s1}{IsDonor}\PY{l+s+s1}{\PYZsq{}}\PY{p}{]}\PY{p}{,} \PY{n}{wam\PYZus{}scores}\PY{o}{=}\PY{n}{scores\PYZus{}wam\PYZus{}1}\PY{p}{,} 
                  \PY{n}{bn\PYZus{}scores}\PY{o}{=}\PY{n}{scores\PYZus{}bn\PYZus{}1}\PY{p}{,} \PY{n}{svm\PYZus{}scores}\PY{o}{=}\PY{n}{scores\PYZus{}svm\PYZus{}1}\PY{p}{)}
\PY{n}{evaltr\PYZus{}2} \PY{o}{=} \PY{n}{Evaluator}\PY{p}{(}\PY{n}{labels}\PY{o}{=}\PY{n}{testing\PYZus{}set2}\PY{p}{[}\PY{l+s+s1}{\PYZsq{}}\PY{l+s+s1}{IsDonor}\PY{l+s+s1}{\PYZsq{}}\PY{p}{]}\PY{p}{,} \PY{n}{wam\PYZus{}scores}\PY{o}{=}\PY{n}{scores\PYZus{}wam\PYZus{}2}\PY{p}{,} 
                  \PY{n}{bn\PYZus{}scores}\PY{o}{=}\PY{n}{scores\PYZus{}bn\PYZus{}2}\PY{p}{,} \PY{n}{svm\PYZus{}scores}\PY{o}{=}\PY{n}{scores\PYZus{}svm\PYZus{}2}\PY{p}{)}
\end{Verbatim}
\end{tcolorbox}

    \begin{tcolorbox}[breakable, size=fbox, boxrule=1pt, pad at break*=1mm,colback=cellbackground, colframe=cellborder]
\prompt{In}{incolor}{ }{\boxspacing}
\begin{Verbatim}[commandchars=\\\{\},fontsize=\small]
\PY{n}{evaltr\PYZus{}1}\PY{o}{.}\PY{n}{Save\PYZus{}figs}\PY{p}{(}\PY{n}{trange}\PY{p}{,} \PY{n}{suffix}\PY{o}{=}\PY{l+s+s1}{\PYZsq{}}\PY{l+s+s1}{1}\PY{l+s+s1}{\PYZsq{}}\PY{p}{)}
\PY{n}{evaltr\PYZus{}2}\PY{o}{.}\PY{n}{Save\PYZus{}figs}\PY{p}{(}\PY{n}{trange}\PY{p}{,} \PY{n}{suffix}\PY{o}{=}\PY{l+s+s1}{\PYZsq{}}\PY{l+s+s1}{2}\PY{l+s+s1}{\PYZsq{}}\PY{p}{)}
\end{Verbatim}
\end{tcolorbox}

    \begin{tcolorbox}[breakable, size=fbox, boxrule=1pt, pad at break*=1mm,colback=cellbackground, colframe=cellborder]
\prompt{In}{incolor}{12}{\boxspacing}
\begin{Verbatim}[commandchars=\\\{\},fontsize=\small]
\PY{n}{evaltr\PYZus{}1}\PY{o}{.}\PY{n}{Cal\PYZus{}metrics}\PY{p}{(}\PY{n}{trange}\PY{p}{)}\PY{o}{.}\PY{n}{to\PYZus{}csv}\PY{p}{(}\PY{l+s+s1}{\PYZsq{}}\PY{l+s+s1}{../Figures/Table\PYZus{}S2.a.csv}\PY{l+s+s1}{\PYZsq{}}\PY{p}{,} \PY{n}{index}\PY{o}{=}\PY{k+kc}{False}\PY{p}{)}
\PY{n}{evaltr\PYZus{}2}\PY{o}{.}\PY{n}{Cal\PYZus{}metrics}\PY{p}{(}\PY{n}{trange}\PY{p}{)}\PY{o}{.}\PY{n}{to\PYZus{}csv}\PY{p}{(}\PY{l+s+s1}{\PYZsq{}}\PY{l+s+s1}{../Figures/Table\PYZus{}S2.b.csv}\PY{l+s+s1}{\PYZsq{}}\PY{p}{,} \PY{n}{index}\PY{o}{=}\PY{k+kc}{False}\PY{p}{)}
\end{Verbatim}
\end{tcolorbox}

    \begin{tcolorbox}[breakable, size=fbox, boxrule=1pt, pad at break*=1mm,colback=cellbackground, colframe=cellborder]
\prompt{In}{incolor}{13}{\boxspacing}
\begin{Verbatim}[commandchars=\\\{\},fontsize=\small]
\PY{n}{fs} \PY{o}{=} \PY{p}{[}\PY{l+s+s1}{\PYZsq{}}\PY{l+s+s1}{../Figures/F1\PYZus{}t\PYZus{}1.png}\PY{l+s+s1}{\PYZsq{}}\PY{p}{,} \PY{l+s+s1}{\PYZsq{}}\PY{l+s+s1}{../Figures/ROC\PYZus{}1.png}\PY{l+s+s1}{\PYZsq{}}\PY{p}{,} \PY{l+s+s1}{\PYZsq{}}\PY{l+s+s1}{../Figures/PRC\PYZus{}1.png}\PY{l+s+s1}{\PYZsq{}}\PY{p}{,} 
      \PY{l+s+s1}{\PYZsq{}}\PY{l+s+s1}{../Figures/F1\PYZus{}t\PYZus{}2.png}\PY{l+s+s1}{\PYZsq{}}\PY{p}{,} \PY{l+s+s1}{\PYZsq{}}\PY{l+s+s1}{../Figures/ROC\PYZus{}2.png}\PY{l+s+s1}{\PYZsq{}}\PY{p}{,} \PY{l+s+s1}{\PYZsq{}}\PY{l+s+s1}{../Figures/PRC\PYZus{}2.png}\PY{l+s+s1}{\PYZsq{}}\PY{p}{]}
\PY{n}{figs} \PY{o}{=} \PY{p}{[}\PY{n}{pli}\PY{o}{.}\PY{n}{imread}\PY{p}{(}\PY{n}{i}\PY{p}{)} \PY{k}{for} \PY{n}{i} \PY{o+ow}{in} \PY{n}{fs}\PY{p}{]}
\PY{n}{plt}\PY{o}{.}\PY{n}{subplots\PYZus{}adjust}\PY{p}{(}\PY{n}{left}\PY{o}{=}\PY{l+m+mf}{0.0}\PY{p}{,}\PY{n}{bottom}\PY{o}{=}\PY{l+m+mf}{0.0}\PY{p}{,}\PY{n}{top}\PY{o}{=}\PY{l+m+mf}{1.2}\PY{p}{,}\PY{n}{right}\PY{o}{=}\PY{l+m+mf}{1.2}\PY{p}{)}
\PY{n}{figure}\PY{p}{,} \PY{p}{(}\PY{p}{(}\PY{n}{ax1}\PY{p}{,} \PY{n}{ax2}\PY{p}{)}\PY{p}{,} \PY{p}{(}\PY{n}{ax3}\PY{p}{,} \PY{n}{ax4}\PY{p}{)}\PY{p}{)} \PY{o}{=} \PY{n}{plt}\PY{o}{.}\PY{n}{subplots}\PY{p}{(}\PY{l+m+mi}{2}\PY{p}{,} \PY{l+m+mi}{2}\PY{p}{,} \PY{n}{dpi}\PY{o}{=}\PY{l+m+mi}{200}\PY{p}{,} \PY{n}{figsize}\PY{o}{=}\PY{p}{(}\PY{l+m+mi}{12}\PY{p}{,}\PY{l+m+mi}{12}\PY{p}{)}\PY{p}{)}
\PY{p}{[}\PY{n}{no\PYZus{}box}\PY{p}{(}\PY{n}{ax}\PY{p}{)} \PY{k}{for} \PY{n}{ax} \PY{o+ow}{in} \PY{p}{(}\PY{n}{ax1}\PY{p}{,} \PY{n}{ax2}\PY{p}{,} \PY{n}{ax3}\PY{p}{,} \PY{n}{ax4}\PY{p}{)}\PY{p}{]}
\PY{n}{ax1}\PY{o}{.}\PY{n}{imshow}\PY{p}{(}\PY{n}{figs}\PY{p}{[}\PY{l+m+mi}{1}\PY{p}{]}\PY{p}{)}\PY{p}{;} \PY{n}{ax2}\PY{o}{.}\PY{n}{imshow}\PY{p}{(}\PY{n}{figs}\PY{p}{[}\PY{l+m+mi}{4}\PY{p}{]}\PY{p}{)}\PY{p}{;} \PY{n}{ax3}\PY{o}{.}\PY{n}{imshow}\PY{p}{(}\PY{n}{figs}\PY{p}{[}\PY{l+m+mi}{2}\PY{p}{]}\PY{p}{)}\PY{p}{;} \PY{n}{ax4}\PY{o}{.}\PY{n}{imshow}\PY{p}{(}\PY{n}{figs}\PY{p}{[}\PY{l+m+mi}{5}\PY{p}{]}\PY{p}{)}
\PY{n}{plt}\PY{o}{.}\PY{n}{tight\PYZus{}layout}\PY{p}{(}\PY{p}{)}
\end{Verbatim}
\end{tcolorbox}

        \begin{figure}
    	\begin{center}
    		\adjustimage{max size={0.9\linewidth}{0.9\paperheight}}{output_33_1.png}
    		\label{mylabel}
    		\caption{Overall comparison for models on balanced data (left) and unbalanced data (right).}
    	\end{center}
    \end{figure}

    \hypertarget{model-selection}{%
\subsection{Model selection}\label{model-selection}}

In the experiments, their performances are slightly different since
their strategies of detecting splice signals are different. But
obviously, there is a situation that unbalanced data affects, compare to
the first experiment, their metrics in the second experiment has
significantly declined, especially \(auPRC\) for WAM and BN.

    \hypertarget{weighted-array-model}{%
\subsubsection{Weighted Array Model}\label{weighted-array-model}}

WAM shows decent accuracy in the experiments considering its low cost of
computating power. It has the ability to discover low level features of
splice site (Fig. 2), but higher level feature detection remains
imposible. As Weighted array model calculate adjacent bases frequencies
at each position, it can only detect common adjacent bases near the
splice sites. In both two experiments, WAM has detected the information
well (see Fig. 1). The result shows that GG, GT, TA bases are more
frequent near splice site. This agrees with some of the facts that have
been found {[}28{]}.

    \begin{tcolorbox}[breakable, size=fbox, boxrule=1pt, pad at break*=1mm,colback=cellbackground, colframe=cellborder]
\prompt{In}{incolor}{14}{\boxspacing}
\begin{Verbatim}[commandchars=\\\{\},fontsize=\small]
\PY{n}{weights\PYZus{}diff} \PY{o}{=} \PY{n}{model\PYZus{}2}\PY{o}{.}\PY{n}{wam}\PY{o}{.}\PY{n}{positive\PYZus{}weights} \PY{o}{\PYZhy{}} \PY{n}{model\PYZus{}2}\PY{o}{.}\PY{n}{wam}\PY{o}{.}\PY{n}{negative\PYZus{}weights}
\PY{n}{plt}\PY{o}{.}\PY{n}{subplots\PYZus{}adjust}\PY{p}{(}\PY{n}{left}\PY{o}{=}\PY{l+m+mf}{0.0}\PY{p}{,}\PY{n}{bottom}\PY{o}{=}\PY{l+m+mf}{0.0}\PY{p}{,}\PY{n}{top}\PY{o}{=}\PY{l+m+mf}{1.2}\PY{p}{,}\PY{n}{right}\PY{o}{=}\PY{l+m+mf}{1.2}\PY{p}{)}
\PY{n}{fig}\PY{p}{,} \PY{n}{ax1} \PY{o}{=} \PY{n}{plt}\PY{o}{.}\PY{n}{subplots}\PY{p}{(}\PY{l+m+mi}{1}\PY{p}{,}\PY{l+m+mi}{1}\PY{p}{,} \PY{n}{figsize}\PY{o}{=}\PY{p}{(}\PY{l+m+mi}{10}\PY{p}{,} \PY{l+m+mi}{9}\PY{p}{)}\PY{p}{,} \PY{n}{dpi}\PY{o}{=}\PY{l+m+mi}{120}\PY{p}{)}
\PY{n}{ax} \PY{o}{=} \PY{n}{sns}\PY{o}{.}\PY{n}{heatmap}\PY{p}{(}\PY{p}{(}\PY{n}{model\PYZus{}1}\PY{o}{.}\PY{n}{wam}\PY{o}{.}\PY{n}{positive\PYZus{}weights} \PY{o}{\PYZhy{}} \PY{n}{model\PYZus{}1}\PY{o}{.}\PY{n}{wam}\PY{o}{.}\PY{n}{negative\PYZus{}weights}\PY{p}{)}\PY{p}{[}\PY{p}{:}\PY{p}{:}\PY{o}{\PYZhy{}}\PY{l+m+mi}{1}\PY{p}{]}\PY{p}{,} 
                 \PY{n}{vmin}\PY{o}{=}\PY{o}{\PYZhy{}}\PY{l+m+mf}{0.5}\PY{p}{,} \PY{n}{vmax}\PY{o}{=}\PY{l+m+mf}{0.5}\PY{p}{,} \PY{n}{ax}\PY{o}{=}\PY{n}{ax1}\PY{p}{,} \PY{n}{cmap}\PY{o}{=}\PY{l+s+s2}{\PYZdq{}}\PY{l+s+s2}{YlGnBu}\PY{l+s+s2}{\PYZdq{}}\PY{p}{)}
\PY{n}{plt}\PY{o}{.}\PY{n}{tight\PYZus{}layout}\PY{p}{(}\PY{p}{)}
\end{Verbatim}
\end{tcolorbox}

    \begin{figure}
    \begin{center}
    \adjustimage{max size={0.9\linewidth}{0.9\paperheight}}{output_36_1.png}
    \label{mylabel}
    \caption{Visualization of the difference in $W$ of WAM  ($W_{donor,2}-W_{pseudo,2}$)\protect\footnotemark.}
    \end{center}
	\end{figure}
\footnotetext{Experiment 2.}
    { \hspace*{\fill} \\}
    
    The result also shows that there is a performance decline in experiment
2 (Fig. 1). The main reason of this is that donor information contained
in second training dataset is much less than first dataset (Fig. 1 \&
3). In other words, the frequency of some adjacent bases within each
position near the donor sites are lower in the unbalanced dataset (Fig.
3). The model cannot learn weights, which were informatively
comprehensive enough, to precisely classify donor site in that situation
(Fig. 1). Thus, To build a reliable WAM model, the information contained
in true splice sites and pseudo sites need to be very comprehensive.

    \begin{tcolorbox}[breakable, size=fbox, boxrule=1pt, pad at break*=1mm,colback=cellbackground, colframe=cellborder]
\prompt{In}{incolor}{15}{\boxspacing}
\begin{Verbatim}[commandchars=\\\{\},fontsize=\small]
\PY{n}{fig}\PY{p}{,} \PY{n}{ax2} \PY{o}{=} \PY{n}{plt}\PY{o}{.}\PY{n}{subplots}\PY{p}{(}\PY{l+m+mi}{1}\PY{p}{,}\PY{l+m+mi}{1}\PY{p}{,} \PY{n}{figsize}\PY{o}{=}\PY{p}{(}\PY{l+m+mi}{10}\PY{p}{,} \PY{l+m+mi}{9}\PY{p}{)}\PY{p}{,} \PY{n}{dpi}\PY{o}{=}\PY{l+m+mi}{120}\PY{p}{)}
\PY{n}{ax} \PY{o}{=} \PY{n}{sns}\PY{o}{.}\PY{n}{heatmap}\PY{p}{(}\PY{p}{(}\PY{p}{(}\PY{n}{model\PYZus{}1}\PY{o}{.}\PY{n}{wam}\PY{o}{.}\PY{n}{positive\PYZus{}weights} \PY{o}{\PYZhy{}} \PY{n}{model\PYZus{}1}\PY{o}{.}\PY{n}{wam}\PY{o}{.}\PY{n}{negative\PYZus{}weights}\PY{p}{)} \PY{o}{\PYZhy{}} 
                  \PY{p}{(}\PY{n}{model\PYZus{}2}\PY{o}{.}\PY{n}{wam}\PY{o}{.}\PY{n}{positive\PYZus{}weights} \PY{o}{\PYZhy{}} \PY{n}{model\PYZus{}2}\PY{o}{.}\PY{n}{wam}\PY{o}{.}\PY{n}{negative\PYZus{}weights}\PY{p}{)}\PY{p}{)}\PY{p}{[}\PY{p}{:}\PY{p}{:}\PY{o}{\PYZhy{}}\PY{l+m+mi}{1}\PY{p}{]}\PY{p}{,} 
                 \PY{n}{vmin}\PY{o}{=}\PY{o}{\PYZhy{}}\PY{l+m+mf}{0.5}\PY{p}{,} \PY{n}{vmax}\PY{o}{=}\PY{l+m+mf}{0.5}\PY{p}{,} \PY{n}{ax}\PY{o}{=}\PY{n}{ax2}\PY{p}{,} \PY{n}{cmap}\PY{o}{=}\PY{l+s+s2}{\PYZdq{}}\PY{l+s+s2}{YlGnBu}\PY{l+s+s2}{\PYZdq{}}\PY{p}{)}
\PY{n}{plt}\PY{o}{.}\PY{n}{tight\PYZus{}layout}\PY{p}{(}\PY{p}{)}
\end{Verbatim}
\end{tcolorbox}
	\begin{figure}
    \begin{center}
    \adjustimage{max size={0.9\linewidth}{0.9\paperheight}}{output_38_0.png}
    \label{mylabel}
    \caption{Visualization of the difference in $D_{exp}$ across experiments ($D_1-D_2$)\protect\footnotemark.}
    \end{center}
	\end{figure}
    \footnotetext{$D_{exp}=W_{donor,exp}-W_{pseudo,exp}$}
        
    \hypertarget{bayesian-network}{%
\subsubsection{Bayesian Network}\label{bayesian-network}}

Bayesian network uses \(\chi^2 -test\) {[}21{]} and conditional
independent test, to build the dependency graph for each position near
the splice site and also donor potential. In both two experiments, BN
has detected the conditional dependence between positions and donor
potential (Fig. 4).

But there are some imperfections. First, directions of edges in the
graph may learn wrong due to separated learning algorithm (Fig. 4).
During structure learning, if there is an edge
\(N_0 \rightarrow N_{-1}\), there will not be an edge
\(N_0 \leftarrow N_{-1}\) because of its DAG structure. But on
choromosome, it is common that adjacent positions affects each other.
One way to address this limitation is to combine the structure learning
and the parameter learning. When there are mutual influences, just
discard edge with less influence. Or when detecting splice signals, I
just wanna keep \(N_{0}\leftarrow N_{-1}\) cause \(N_0\) is the site.
Second, BN's learning algorithm utilizes \(\chi^2-test\) and conditional
independent test and so more sensitive to unbalanced data (Fig. 4)
{[}14{]}. So there is a significant performance decline for BN compare
to SVM (Fig. 1). Furthermore, BN is susceptible to not only information
contained in dataset, but also distribution of data compares with WAM.
This would explain the fact that performance declines in experiment 2.

Finally, classical integrated BN is in a position to detect not only
dependence between positions, but also dependences between positions and
donor potential, because it learns parameters from all data, donor sites
and pseudo sites. While expanded Bayesian networks {[}22{]} can just
detect dependence between positions, because it learns positive and
negative BNs' parameters using positive and negative data respectively.

    \begin{tcolorbox}[breakable, size=fbox, boxrule=1pt, pad at break*=1mm,colback=cellbackground, colframe=cellborder]
\prompt{In}{incolor}{16}{\boxspacing}
\begin{Verbatim}[commandchars=\\\{\},fontsize=\small]
\PY{n}{fig}\PY{p}{,} \PY{p}{(}\PY{n}{ax1}\PY{p}{,} \PY{n}{ax2}\PY{p}{)} \PY{o}{=} \PY{n}{plt}\PY{o}{.}\PY{n}{subplots}\PY{p}{(}\PY{l+m+mi}{1}\PY{p}{,} \PY{l+m+mi}{2}\PY{p}{,} \PY{n}{figsize}\PY{o}{=}\PY{p}{(}\PY{l+m+mi}{10}\PY{p}{,}\PY{l+m+mi}{5}\PY{p}{)}\PY{p}{,} \PY{n}{dpi}\PY{o}{=}\PY{l+m+mi}{120}\PY{p}{)}
\PY{n}{G1} \PY{o}{=} \PY{n}{model\PYZus{}1}\PY{o}{.}\PY{n}{bn}\PY{o}{.}\PY{n}{networks}\PY{p}{;} \PY{n}{bicolors} \PY{o}{=} \PY{p}{[}\PY{l+s+s1}{\PYZsq{}}\PY{l+s+s1}{deepskyblue}\PY{l+s+s1}{\PYZsq{}}\PY{p}{,}\PY{l+s+s1}{\PYZsq{}}\PY{l+s+s1}{salmon}\PY{l+s+s1}{\PYZsq{}}\PY{p}{]}
\PY{n}{nodesize} \PY{o}{=} \PY{n}{np}\PY{o}{.}\PY{n}{array}\PY{p}{(}\PY{p}{[}\PY{n+nb}{float}\PY{p}{(}\PY{n}{G1}\PY{o}{.}\PY{n}{degree}\PY{p}{(}\PY{n}{v}\PY{p}{)} \PY{o}{+} \PY{l+m+mi}{1}\PY{p}{)} \PY{k}{for} \PY{n}{v} \PY{o+ow}{in} \PY{n}{G1}\PY{p}{]}\PY{p}{)}
\PY{n}{nodecolor} \PY{o}{=} \PY{n}{np}\PY{o}{.}\PY{n}{array}\PY{p}{(}\PY{p}{[}\PY{n}{bicolors}\PY{p}{[}\PY{n+nb}{int}\PY{p}{(}\PY{l+s+s1}{\PYZsq{}}\PY{l+s+s1}{L}\PY{l+s+s1}{\PYZsq{}} \PY{o}{==} \PY{n}{i}\PY{p}{)}\PY{p}{]} \PY{k}{for} \PY{n}{i} \PY{o+ow}{in} \PY{n}{G1}\PY{p}{]}\PY{p}{)}
\PY{n}{nx}\PY{o}{.}\PY{n}{draw}\PY{p}{(}\PY{n}{G1}\PY{p}{,} \PY{n}{with\PYZus{}labels}\PY{o}{=}\PY{k+kc}{True}\PY{p}{,} \PY{n}{textsize}\PY{o}{=}\PY{l+m+mi}{10000}\PY{p}{,} \PY{n}{node\PYZus{}color}\PY{o}{=}\PY{n}{nodecolor}\PY{p}{,}
        \PY{n}{pos}\PY{o}{=}\PY{n}{nx}\PY{o}{.}\PY{n}{nx\PYZus{}agraph}\PY{o}{.}\PY{n}{graphviz\PYZus{}layout}\PY{p}{(}\PY{n}{G1}\PY{p}{,} \PY{n}{prog}\PY{o}{=}\PY{l+s+s1}{\PYZsq{}}\PY{l+s+s1}{neato}\PY{l+s+s1}{\PYZsq{}}\PY{p}{)}\PY{p}{,} \PY{n}{ax}\PY{o}{=}\PY{n}{ax1}\PY{p}{,}               
        \PY{n}{node\PYZus{}size}\PY{o}{=}\PY{n}{nodesize}\PY{o}{*}\PY{l+m+mi}{150}\PY{p}{)}
\PY{n}{G2} \PY{o}{=} \PY{n}{model\PYZus{}2}\PY{o}{.}\PY{n}{bn}\PY{o}{.}\PY{n}{networks}\PY{p}{;} \PY{n}{bicolors} \PY{o}{=} \PY{p}{[}\PY{l+s+s1}{\PYZsq{}}\PY{l+s+s1}{deepskyblue}\PY{l+s+s1}{\PYZsq{}}\PY{p}{,}\PY{l+s+s1}{\PYZsq{}}\PY{l+s+s1}{salmon}\PY{l+s+s1}{\PYZsq{}}\PY{p}{]}
\PY{n}{nodesize} \PY{o}{=} \PY{n}{np}\PY{o}{.}\PY{n}{array}\PY{p}{(}\PY{p}{[}\PY{n+nb}{float}\PY{p}{(}\PY{n}{G2}\PY{o}{.}\PY{n}{degree}\PY{p}{(}\PY{n}{v}\PY{p}{)} \PY{o}{+} \PY{l+m+mi}{1}\PY{p}{)} \PY{k}{for} \PY{n}{v} \PY{o+ow}{in} \PY{n}{G2}\PY{p}{]}\PY{p}{)}
\PY{n}{nodecolor} \PY{o}{=} \PY{n}{np}\PY{o}{.}\PY{n}{array}\PY{p}{(}\PY{p}{[}\PY{n}{bicolors}\PY{p}{[}\PY{n+nb}{int}\PY{p}{(}\PY{l+s+s1}{\PYZsq{}}\PY{l+s+s1}{L}\PY{l+s+s1}{\PYZsq{}} \PY{o}{==} \PY{n}{i}\PY{p}{)}\PY{p}{]} \PY{k}{for} \PY{n}{i} \PY{o+ow}{in} \PY{n}{G2}\PY{p}{]}\PY{p}{)}
\PY{n}{nx}\PY{o}{.}\PY{n}{draw}\PY{p}{(}\PY{n}{G2}\PY{p}{,} \PY{n}{with\PYZus{}labels}\PY{o}{=}\PY{k+kc}{True}\PY{p}{,} \PY{n}{textsize}\PY{o}{=}\PY{l+m+mi}{10000}\PY{p}{,} \PY{n}{node\PYZus{}color}\PY{o}{=}\PY{n}{nodecolor}\PY{p}{,}
        \PY{n}{pos}\PY{o}{=}\PY{n}{nx}\PY{o}{.}\PY{n}{nx\PYZus{}agraph}\PY{o}{.}\PY{n}{graphviz\PYZus{}layout}\PY{p}{(}\PY{n}{G2}\PY{p}{,} \PY{n}{prog}\PY{o}{=}\PY{l+s+s1}{\PYZsq{}}\PY{l+s+s1}{neato}\PY{l+s+s1}{\PYZsq{}}\PY{p}{)}\PY{p}{,} \PY{n}{ax}\PY{o}{=}\PY{n}{ax2}\PY{p}{,}
        \PY{n}{node\PYZus{}size}\PY{o}{=}\PY{n}{nodesize}\PY{o}{*}\PY{l+m+mi}{150}\PY{p}{)}
\end{Verbatim}
\end{tcolorbox}
	\begin{figure}
	\begin{center}
		\adjustimage{max size={0.6\linewidth}{0.9\paperheight}}{output_40_0.png}
	\end{center}
	\caption{Graph for Bayesian network. $G_1$ (left) and $G_2$ (right).}
	\end{figure}
    
    \hypertarget{support-vector-machine}{%
\subsubsection{Support Vector Machine}\label{support-vector-machine}}

Support Vector machines have performed best and nearly the same in the
experiments. The result shows that SVM has the potential to take care of
unbalanced data. There are several benefits using SVM.

First, SVM was appropriate to provide for classification. It has the
highest accuracy amoung all three models used in the experiments. The
margin maximal and kernel trick gives it ability to classify very
complex data. One more important feature of SVM is that only support
vectors are important in order to classification. The rest can be
ignored. This can easily reduce the computing power needed.

Second, when dealing with unbalanced data, the number of support vectors
can be slightly the same for each class (Table S1). This makes its
accuracy slightly the save as first experiment (Fig. 1). It handled the
unbalanced data elegantly.

Third. There are several mature optimization algorithms with low time
complexity and sophisticated software tools {[}11,23,25{]}. They were
nicely implemented, make SVM's training and prediction fast. However,
Visualization of the hyper-plane and high-dimensional space is
impossible. Although its accuracy is excellent. It still likes a black
box. We cannot know which feature are the most important or what kind of
the signals were detected.

    \begin{tcolorbox}[breakable, size=fbox, boxrule=1pt, pad at break*=1mm,colback=cellbackground, colframe=cellborder]
\prompt{In}{incolor}{22}{\boxspacing}
\begin{Verbatim}[commandchars=\\\{\},fontsize=\small]
\PY{n}{df} \PY{o}{=} \PY{n}{pd}\PY{o}{.}\PY{n}{DataFrame}\PY{p}{(} \PY{p}{(}\PY{n}{model\PYZus{}1}\PY{o}{.}\PY{n}{svm}\PY{o}{.}\PY{n}{n\PYZus{}support\PYZus{}}\PY{p}{,} \PY{n}{model\PYZus{}2}\PY{o}{.}\PY{n}{svm}\PY{o}{.}\PY{n}{n\PYZus{}support\PYZus{}}\PY{p}{)}\PY{p}{,} \PY{n}{columns}\PY{o}{=}\PY{p}{[}\PY{l+s+s1}{\PYZsq{}}\PY{l+s+s1}{Donor}\PY{l+s+s1}{\PYZsq{}}\PY{p}{,}\PY{l+s+s1}{\PYZsq{}}\PY{l+s+s1}{Pseudo}\PY{l+s+s1}{\PYZsq{}}\PY{p}{]}\PY{p}{)}
\PY{n}{df}\PY{o}{.}\PY{n}{index} \PY{o}{=} \PY{p}{[}\PY{l+s+s1}{\PYZsq{}}\PY{l+s+s1}{SVM 1}\PY{l+s+s1}{\PYZsq{}}\PY{p}{,} \PY{l+s+s1}{\PYZsq{}}\PY{l+s+s1}{SVM 2}\PY{l+s+s1}{\PYZsq{}}\PY{p}{]}\PY{p}{;}\PY{n+nb}{print}\PY{p}{(}\PY{n}{df}\PY{o}{.}\PY{n}{to\PYZus{}latex}\PY{p}{(}\PY{p}{)}\PY{p}{)}
\end{Verbatim}
\end{tcolorbox}

%\appendix      
\setcounter{table}{0}   %从零开始编号
\renewcommand{\thetable}{S\arabic{table}}

\begin{table}[h]
\centering
	\caption{Number of support vectors in SVM}
	\begin{threeparttable}
	\begin{tabular}{lcc}
	\toprule
	{} &  Donor &  Pseudo \\
	\midrule
	SVM 1 &    350 &     352 \\
	SVM 2 &    543 &     509 \\
	\bottomrule
\end{tabular}
\end{threeparttable}
\end{table}

    \hypertarget{metrics}{%
\subsection{Metrics}\label{metrics}}

In the experiments, we used
\(TPR, FPR, Sn, Sp, Rc, Pr, F1-score, auROC, auPRC\) metrics to evaluate
performances of models. The result shows that \(auPRC\) is more
sensitive to unbalanced data than \(auROC\) (Fig. 1).

\hypertarget{simple-mathamatical-formulation}{%
\paragraph{Simple mathamatical
formulation:}\label{simple-mathamatical-formulation}}

Here we note \(D=Num_p(TP,FN)-Num_n(FP,TN)\) as the difference between
two classes. \(TPR(Num_p)\), \(Sn(Num_p)\) and \(Rc(Num_p)\) are
independent with \(Num_n\), and \(FPR(Num_n), Sp(Num_n)\) are
independent on \(Num_p\), so they are independent with \(D\). \(Pr\) is
the only metric that is dependent on \(D\). So
\(auPRC=\sum_{i=1}^{N-1}\frac{(Rc_{i+1} - Rc_i)(Pr_i+Pr_{i+1})}{2}\) can
enlarge the difference of performance when comparing different models on
unbalanced data.

When comparing models on an unbalanced dataset, we should not miss
\(Pr\) and \(auPRC\). Besides, \(F1-score=\frac{2Pr Rc}{Pr+Rc}\) also
has ability to enlarge this kind of difference.

    \hypertarget{threshold-selection}{%
\subsection{Threshold selection}\label{threshold-selection}}

\(F1-score\) {[}24{]} is another important metric. Since Pr and Rc are
paradoxical sometimes, we cannot have highest \(Rc\) and \(Pr\) at the
same time. When selection thresholds, we use \(F1-score\) instead.

The result shows that all \(F1-score\) reach its peak at \(t\approx 0\).
So \(t=0\) is a suitable threshold for 3 models. For other metrics
involved in threshold selection, see Table
\href{https://github.com/AdeBC/GSSR/blob/master/Figures/Table_S2.a.csv}{S2.a}
and
\href{https://github.com/AdeBC/GSSR/blob/master/Figures/Table_S2.b.csv}{S2.b}.

    \begin{tcolorbox}[breakable, size=fbox, boxrule=1pt, pad at break*=1mm,colback=cellbackground, colframe=cellborder]
\prompt{In}{incolor}{18}{\boxspacing}
\begin{Verbatim}[commandchars=\\\{\},fontsize=\small]
\PY{n}{fs} \PY{o}{=} \PY{p}{[}\PY{l+s+s1}{\PYZsq{}}\PY{l+s+s1}{../Figures/F1\PYZus{}t\PYZus{}1.png}\PY{l+s+s1}{\PYZsq{}}\PY{p}{,} \PY{l+s+s1}{\PYZsq{}}\PY{l+s+s1}{../Figures/F1\PYZus{}t\PYZus{}2.png}\PY{l+s+s1}{\PYZsq{}}\PY{p}{]}
\PY{n}{figs} \PY{o}{=} \PY{p}{[}\PY{n}{pli}\PY{o}{.}\PY{n}{imread}\PY{p}{(}\PY{n}{i}\PY{p}{)} \PY{k}{for} \PY{n}{i} \PY{o+ow}{in} \PY{n}{fs}\PY{p}{]}
\PY{n}{figure}\PY{p}{,} \PY{p}{(}\PY{n}{ax1}\PY{p}{,} \PY{n}{ax2}\PY{p}{)} \PY{o}{=} \PY{n}{plt}\PY{o}{.}\PY{n}{subplots}\PY{p}{(}\PY{l+m+mi}{2}\PY{p}{,} \PY{l+m+mi}{1}\PY{p}{,} \PY{n}{dpi}\PY{o}{=}\PY{l+m+mi}{200}\PY{p}{,} \PY{n}{figsize}\PY{o}{=}\PY{p}{(}\PY{l+m+mi}{12}\PY{p}{,}\PY{l+m+mi}{12}\PY{p}{)}\PY{p}{)}
\PY{n}{plt}\PY{o}{.}\PY{n}{subplots\PYZus{}adjust}\PY{p}{(}\PY{n}{left}\PY{o}{=}\PY{l+m+mf}{0.0}\PY{p}{,}\PY{n}{bottom}\PY{o}{=}\PY{l+m+mf}{0.0}\PY{p}{,}\PY{n}{top}\PY{o}{=}\PY{l+m+mi}{1}\PY{p}{,}\PY{n}{right}\PY{o}{=}\PY{l+m+mi}{1}\PY{p}{)}\PY{p}{;} 
\PY{n}{no\PYZus{}box}\PY{p}{(}\PY{n}{ax1}\PY{p}{)}\PY{p}{;} \PY{n}{ax1}\PY{o}{.}\PY{n}{imshow}\PY{p}{(}\PY{n}{figs}\PY{p}{[}\PY{l+m+mi}{0}\PY{p}{]}\PY{p}{)}\PY{p}{;}\PY{n}{no\PYZus{}box}\PY{p}{(}\PY{n}{ax2}\PY{p}{)}\PY{p}{;} \PY{n}{ax2}\PY{o}{.}\PY{n}{imshow}\PY{p}{(}\PY{n}{figs}\PY{p}{[}\PY{l+m+mi}{1}\PY{p}{]}\PY{p}{)}\PY{p}{;}\PY{n}{plt}\PY{o}{.}\PY{n}{tight\PYZus{}layout}\PY{p}{(}\PY{p}{)}
\end{Verbatim}
\end{tcolorbox}
	
	\begin{figure}
    \begin{center}
    \adjustimage{max size={0.8\linewidth}{0.9\paperheight}}{output_45_0.png}
    \caption{Visualization for F1-score \& Threshold across experiments $exp_1$ (top) and $exp_2$ (bottom).}
    \end{center}
	\end{figure}
    
    \hypertarget{conclusion-and-prospect}{%
\section{Conclusion and Prospect}\label{conclusion-and-prospect}}

    Some facts have been found after these attempts. First, \textbf{Weighted
Array Model} is good at detecting dependence between adjacent positions
and costs lowest computing power. Second, \textbf{Bayesian network}
still suffers from learning algorithm. PC-Contraint based algorithm and
MMHC algorithm still not perfect and are so time consuming. It takes me
10 hours to train a Bayesian network. Third, \textbf{Support Vector
Machine} is very accurate and fast when dealing with small dataset.
Finally, when dataset is unbalanced, A very wise choice would be
selecting not only \(auROC\) but also \(auPRC\) as metrics to evaluate
models. Or make dataset balanced before training, some methods have
been proposed to complete this.

The Pros and Cons of models are arranged below.

\begin{longtable}[]{@{}lll@{}}
\toprule
\begin{minipage}[b]{0.35\columnwidth}\raggedright
Model\strut
\end{minipage} & \begin{minipage}[b]{0.28\columnwidth}\raggedright
Pros\strut
\end{minipage} & \begin{minipage}[b]{0.28\columnwidth}\raggedright
Cons\strut
\end{minipage}\tabularnewline
\midrule
\endhead
\begin{minipage}[t]{0.35\columnwidth}\raggedright
\textbf{WAM}\strut
\end{minipage} & \begin{minipage}[t]{0.28\columnwidth}\raggedright
Low cost of computing power, Splice signal detection, Easy to
visualize\strut
\end{minipage} & \begin{minipage}[t]{0.28\columnwidth}\raggedright
Accuracy not perfect\strut
\end{minipage}\tabularnewline
\begin{minipage}[t]{0.35\columnwidth}\raggedright
\textbf{BN}\strut
\end{minipage} & \begin{minipage}[t]{0.28\columnwidth}\raggedright
Easy to visualize, Causual relationship inference, Splice signal
detection\strut
\end{minipage} & \begin{minipage}[t]{0.28\columnwidth}\raggedright
Immature learning algorithms, accuracy still not perfect\strut
\end{minipage}\tabularnewline
\begin{minipage}[t]{0.35\columnwidth}\raggedright
\textbf{SVM}\strut
\end{minipage} & \begin{minipage}[t]{0.28\columnwidth}\raggedright
Extreamely high accuracy, Fast and mature learning algotithm
{[}23,25{]}\strut
\end{minipage} & \begin{minipage}[t]{0.28\columnwidth}\raggedright
Cannot visualize, Black box\strut
\end{minipage}\tabularnewline
\bottomrule
\end{longtable}

    \hypertarget{limitations-and-prospect}{%
\subsection{Limitations and Prospect}\label{limitations-and-prospect}}

There are still some limitations about this work. First, the information
contained in datasets are different, this may cause a little accuracy
decline. Second, the dataset used in the experiments are still too
small, only \(5000\) samples are used to training models.

Corresponding with limitations, there are two experiments need to be
done in the future. First, make information-equal datasets to repeat
the experiments. Second, experiments on larger datasets need to be done
as the proportion of donor sites is near \(10^{-3}\) or even smaller.

    \hypertarget{availability}{%
\section{Availability}\label{availability}}

Dataset was provided by Prof.~Zhou of Huazhong University of Science
and Technology. Source code of this project are hosted on
\href{https://github.com/AdeBC/GSSR}{Github}. Besides, to encourage
others to reproduce this project, all source code involved in this
report was remained. Anyone can configure dependence environment can
easily reproduce it.

Making research projects more reproducible is highly recommended here.

    \hypertarget{acknowledgements}{%
\section{Acknowledgements}\label{acknowledgements}}

First of all, I would like to extend my sincere gratitude to Professor
Yanhong Zhou, for his valuable knowledge and experience in bioinformatic
data mining that I've gained. I am deeply grateful of his patient
teaching during the COVID-19 epidemic.

Second, I am also deeply indebted to all the other researchers and geeks
in Bioinformatics and Data Mining for their great works and wonderful
programming experiences and skills shared on the web. I cannot finish
this project without them.

Finally, Special thanks should go to my parents, girlfriend and also
other friends, for their continuous support and encouragement.

    \hypertarget{references}{%
\section{References}\label{references}}

\begin{enumerate}
\def\labelenumi{\arabic{enumi}.}
\tightlist
\item
  M.O. Zhang, T.G. Marr, A weight array method for splicing signal
  analysis, Bioinformatics, Volume 9, Issue 5, October 1993, Pages
  499--509, https://doi.org/10.1093/bioinformatics/9.5.499
\item
  Sinha, R., Hiller, M., Pudimat, R. et al.~Improved identification of
  conserved cassette exons using Bayesian networks. BMC Bioinformatics
  9, 477 (2008). https://doi.org/10.1186/1471-2105-9-477
\item
  Pearl J. Bayesian netwcrks: A model cf self-activated memory for
  evidential reasoning{[}C{]}//Proceedings of the 7th Conference of the
  Cognitive Science Society, University of California, Irvine, CA, USA.
  1985: 15-17.
\item
  Sonnenburg, S., Schweikert, G., Philips, P. et al.~Accurate splice
  site prediction using support vector machines. BMC Bioinformatics 8,
  S7 (2007). https://doi.org/10.1186/1471-2105-8-S10-S7
\item
  Cortes, C., Vapnik, V. Support-vector networks. Mach Learn 20,
  273--297 (1995). https://doi.org/10.1007/BF00994018
\item
  Boyd K, Eng K H, Page C D. Area under the precision-recall curve:
  point estimates and confidence intervals{[}C{]}//Joint European
  conference on machine learning and knowledge discovery in databases.
  Springer, Berlin, Heidelberg, 2013: 451-466.
\item
  Gilbert W. Why genes in pieces?. \emph{Nature}. 1978;271(5645):501.
  doi:10.1038/271501a0
\item
  deeplizard. One-hot Encoding explained. YouTube. 2018-01-12
  {[}2019-09-08{]}
\item
  Nomenclature for Incompletely Specified Bases in Nucleic Acid
  Sequences, NC-IUB, 1984.
\item
  Oliphant T E. Python for scientific computing{[}J{]}. Computing in
  Science \& Engineering, 2007, 9(3): 10-20.
\item
  Pedregosa F, Varoquaux G, Gramfort A, et al.~Scikit-learn: Machine
  learning in Python{[}J{]}. the Journal of machine Learning research,
  2011, 12: 2825-2830.
\item
  Ankan A, Panda A. pgmpy: Probabilistic graphical models using
  python{[}C{]}//Proceedings of the 14th Python in Science Conference
  (SCIPY 2015). 2015.
\item
  C.D. Manning, P. Raghavan and M. Schütze (2008). \emph{Introduction to
  Information Retrieval}. Cambridge University Press, p.~260.
\item
  P. Spirtes, C. Glymour, and R. Scheines. Causation, prediction, and
  search. Springer-Verlag, New York, 1993.
\item
  Vert J P, Tsuda K, Schölkopf B. A primer on kernel methods{[}J{]}.
  Kernel methods in computational biology, 2004, 47: 35-70.
\item
  Stone M. Cross‐validatory choice and assessment of statistical
  predictions{[}J{]}. Journal of the Royal Statistical Society: Series B
  (Methodological), 1974, 36(2): 111-133.
\item
  Fawcett, Tom (2006); \emph{An introduction to ROC analysis}, Pattern
  Recognition Letters, 27, 861--874.
\item
  Boyd K, Eng K H, Page C D. Area under the precision-recall curve:
  point estimates and confidence intervals{[}C{]}//Joint European
  conference on machine learning and knowledge discovery in databases.
  Springer, Berlin, Heidelberg, 2013: 451-466.
\item
  Hanley JA, McNeil BJ. A method of comparing the areas under receiver
  operating characteristic curves derived from the same cases.
  \emph{Radiology}. 1983;148(3):839-843.
  doi:10.1148/radiology.148.3.6878708
\item
  Ossendrijver M. Ancient Babylonian astronomers calculated Jupiter's
  position from the area under a time-velocity graph. \emph{Science}.
  2016;351(6272):482-484. doi:10.1126/science.aad8085
\item
  McHugh M L. The chi-square test of independence{[}J{]}. Biochemia
  medica: Biochemia medica, 2013, 23(2): 143-149.
\item
  Chen T M, Lu C C, Li W H. Prediction of splice sites with dependency
  graphs and their expanded bayesian networks{[}J{]}. Bioinformatics,
  2005, 21(4): 471-482.
\item
  Platt J. Sequential minimal optimization: A fast algorithm for
  training support vector machines{[}J{]}. 1998.
\item
  Sasaki, Yutaka. (2007). The truth of the F-measure. Teach Tutor Mater.
\item
  Chang C C, Lin C J. LIBSVM: A library for support vector
  machines{[}J{]}. ACM transactions on intelligent systems and
  technology (TIST), 2011, 2(3): 1-27.
\item
  J. D. Hunter, ``Matplotlib: A 2D Graphics Environment'', Computing in
  Science \& Engineering, vol.~9, no. 3, pp.~90-95, 2007
\item
  Waskom, M. et al., 2017. \emph{mwaskom/seaborn: v0.8.1 (September
  2017)}, Zenodo. Available at: https://doi.org/10.5281/zenodo.883859.
\item
  Wikipedia contributors. (2020, June 17). RNA splicing. In
  \emph{Wikipedia, The Free Encyclopedia}. Retrieved 01:09, June 20,
  2020, from
  https://en.wikipedia.org/w/index.php?title=RNA\_splicing\&oldid=962973627
\item
  Compeau P, Pevzner P A. Bioinformatics Algorithms: An Active Learning
  Approach. La Jolla{[}M{]}. CA: Active Learning Publishers, 2018.
\item
  Li H, Cabeli V, Sella N, et al.~Constraint-based causal structure
  learning with consistent separating sets{[}C{]}//Advances in Neural
  Information Processing Systems. 2019: 14257-14266.
\end{enumerate}

    \begin{tcolorbox}[breakable, size=fbox, boxrule=1pt, pad at break*=1mm,colback=cellbackground, colframe=cellborder]
\prompt{In}{incolor}{ }{\boxspacing}
\begin{Verbatim}[commandchars=\\\{\},fontsize=\small]

\end{Verbatim}
\end{tcolorbox}


    % Add a bibliography block to the postdoc
    
    
    
\end{document}
